\documentclass[12pt]{scrartcl}

\usepackage[german]{babel}

\usepackage{FreeDoko}

\title{Einfache Doppelkopf-Spielregeln}
\author{Borg Enders, Dr. Diether Knof}
\date{15.~Dezember 2011}

\begin{document}

\maketitle

\noindent
Dies ist eine vereinfachte Fassung der Turnierspielregeln des Deutschen Doppelkopf Verbandes e.V. (DDV) für den täglichen Gebrauch und zur Einweisung von Spielern, die das Spiel grundsätzlich kennen, in die Spielregeln des DDV.

\tableofcontents

\section{Das Doppelkopfblatt}
Dieses besteht aus je 12 Karten der vier Farben Kreuz (\kreuz), Pik (\pik), Herz (\herz) und Karo (\karo), nämlich (Zählwert in Klammern) pro Farbe je zwei Asse (11), zwei Zehnen (10), zwei Könige (4), zwei Damen (3), zwei
Buben (2) und zwei Neunen (0) (es wird also mit Neunen gespielt).

\section{Spieler und Parteien}
Doppelkopf wird mit vier Spielern gespielt (bei fünf Spielern setzt der Geber aus). Im Normalspiel spielen die Spieler, die die \kreuz-Damen besitzen (Re-Patei) gegen die beiden anderen (Kontra-Partei). Es ist allerdings auch möglich, daß ein Spieler gegen die drei anderen ein sogenanntes Solo spielt.

\section{Trumpf und Fehl}
Grundsätzlich besteht Bedienpflicht. Kann eine angespielte Karte nicht bedient werden, kann getrumpft oder abgeworfen werden.

\section{Normalspiel}
Hier sind die folgenden Karten in der angegebenen Reihenfolge Trumpf: \herz Zehn, \kreuz Dame, \pik Dame, \herz Dame, \karo Dame, \kreuz Bube, \pik Bube, \herz Bube, \karo Bube, \karo As, \karo Zehn, \karo König, \karo Neun. Da alle Karten zweimal vorhanden sind, gibt es 26 Trumpf im Spiel. Alle anderen Karten sind Fehl in der Reihenfolge: As, Zehn, König, Neun. Es sind also insgesamt 22 Fehlkarten im Spiel (man beachte die Sonderstellung der \herz Zehn!).

\section{Hochzeit}
Besitzt ein Spieler beide \kreuz -Damen, so hat er eine "`Hochzeit"'. Sein Partner wird derjenige, der innerhalb der ersten drei Stiche den ersten Stich macht, der nicht an den Hochzeiter geht (Klärungsstich). Falls eine Hochzeit nicht angemeldet wird oder der Hochzeiter die ersten drei Stiche macht, dann spielt der Hochzeiter ein Farbsolo in \karo.

\section{Solospiele}
Ein Solospiel ist ein Spiel bei dem ein Spieler alleine gegen die drei anderen Spieler spielt.

\subsection{Damensolo}
Alle Damen sind Trumpf in der Reihenfolge \kreuz, \pik, \herz, \karo. Es gibt also acht Trumpf. Alle anderen Karten sind Fehl in der Reihenfolge As, Zehn, König, Bube, Neun.

\subsection{Bubensolo}
Alle Buben sind Trumpf in der Reihenfolge \kreuz, \pik, \herz, \karo. Es gibt also acht Trumpf. Alle anderen Karten sind Fehl in der Reihenfolge As, Zehn, König, Dame, Neun.

\subsection{Farbsolo}
\herz Zehnen, alle Damen und alle Buben sind wie im Normalspiel. \karo As, \karo Zehn, \karo König und \karo Neun werden durch die entsprechenden Karten der gewählten Trumpffarbe (\kreuz, \pik, \herz oder \karo) ersetzt. Wichtig: Beim \herz -Solo bleibt die \herz Zehnen hoch! Es sind dann zwei Trumpf weniger im Spiel.

\subsection{As-Solo (Fleischloser)}
Es gibt keine Trümpfe. Die Karten gelten in der Reihenfolge As, Zehn, König, Dame, Bube, Neun.

\section{Spielverlauf} Nach dem
Mischen besteht Abhebepflicht. Es werden dann viermal drei Karten an jeden Spieler verteilt. Anschließend erfolgt die Vorbehaltsfrage:

Zwei Vorbehalte sind möglich : "`Solo"' oder "`Hochzeit"'. Zunächst stellt der links vom Geber Sitzende fest, ob er einen Vorbehalt hat (dann sagt er: "`Vorbehalt"') oder nicht (dann sagt er: "`gesund"'). Es folgt der von ihm links Sitzende usw., so daß der Geber als Letzter meldet. Melden alle Spieler "`gesund"', spielt der links vom Geber sitzende zum Normalspiel auf. Melden ein oder mehrere Spieler "`Vorbehalt"', dann geben sie jetzt in der gleichen Reihenfolge ihren Vorbehalt bekannt und zwar nur durch "`Solo"' oder "`Hochzeit"'. Die Vorbehaltsfrage endet, sobald ein Spieler den gefragten Vorbehalt bejaht. Danach benennt dieser Spieler den Spieltyp (tauft Vorbehalt). Ein Solo ist gegenüber einer Hochzeit der höherrangige Vorbehalt. Bei mehreren Soli erhält der am weitesten vorne sitzende Spieler das Spielrecht. Das Aufspiel bleibt dabei grundsätzlich unberührt.

\section{An- und Absage}
Mit einer Ansage oder Absage bekundet ein Spieler die Ansicht, daß ein Spiel seiner Meinung nach auf eine bestimmte Weise enden wird und erhöht hierdurch die Punkte die seine Partei in diesem Spiel erzielen können.

\subsection{Ansage}
Mit der Ansage "`Re"' (Re-Partei) oder "`Kontra"' (Kontra-Partei) zeigt der Anasgende an, daß er glaubt, zusammen mit seinem Partner das Spiel gewinnen zu können (Vorsicht: Es gibt auch unzählige "`taktische"' Gründe für eine Ansage!). Eine Ansage ist solange möglich, wie der Ansagende noch 11 Karten auf der Hand hat, jedoch nicht vor Beendigung der Vorbehaltsfrage. Bei einer Hochzeit darf keine Ansage gemacht werden, bis der Klärungsstich beendet ist. Falls die Klärung mit dem ersten Stich erfolgt, müsssen zur Ansage noch 11 Karten auf der Hand sein, wenn mit dem zweiten noch Zehn und wenn mit dem dritten noch neun.

\subsection{Absagen}
Ausschließlich nach erfolgter Ansage können die Partner durch Absagen den Wert des Spieles erhöhen.
\begin{table}[htbp]
  \centering
  \begin{tabular}{ll}
    "`Keine 90"' & mit 10 (9, 8) Karten auf der Hand\\
    "`Keine 60"' & mit 9 (8, 7) Karten auf der Hand\\
    "`Keine 30"' & mit 8 (7, 6) Karten auf der Hand\\
    "`Schwarz"' & mit 7 (6, 5) Karten auf der Hand
  \end{tabular}
  \caption{Absagen}
\end{table}
Die Angaben in Klammern gelten für die Hochzeit mit Klärung im zweiten bzw. dritten Stich. Die Absage bedeutet, daß die Gegenpartei keine 90 (60, 30 oder 0) Punkte erreichen wird. Absagen können auch frühzeitig erfolgen. Es darf aber keine ausgelassen werden. Auf eine Ansage oder Absage kann mit jeweils einer Karte weniger auf der Hand als für die An-/Absage erforderlich ist "`Re"' oder "`Kontra"' erwidert werden.

\section{Spielende und Wertung}
Ein einzelnes Spiel endet in der Regel nach dem Spielen aller 12 Karten und es werden dann die Punkte der Gewinnerpartei für dieses Spiel ermittelt.

\subsection{Spielende}
Es sind insgesamt 240 Punkte im Spiel. Die Re-Partei hat gewonnen, wenn sie 121 (151, 181, 211) Punkte (oder alle Stiche) gesammelt hat, außer es ist nur "`Kontra"' ohne "`Re"' angesagt worden. Dann reichen 120 (151, \dots) gesammelte Punkte. Die Kontra-Partei gewinnt mit 120 (151, 181, 211 oder allen Stichen) gesammelten Punkten, außer sie hat "`Kontra"' ohne "`Re"' angesagt, dann benötigt sie 121 (151, \dots) gesammelte Punkte. Wenn "`Re"' und "`Kontra"' angesagt wurden reichen der Kontra-Partei 120 Punkte für den Sieg.

\subsection{Wertung}
Es wird nach der Plus-Minus-Wertung gewertet: Im Normalspiel erhalten die Spieler der Siegerpartei folgende Spielpunkte mit positivem, die Spieler der Verliererpartei mit negativem Vorzeichen.
\begin{table}[htbp]
  \centering
  \begin{tabular}{lll}
    1. & Gewonnen & 1 Punkt \\
    2. & Ansage (Kontra/Re) & 2 Punkte\\
    3. & unter 90/60/30/schwarz gespielt & jeweils 1 Punkt\\
    4. & keine 90/60/30/schwarz abgesagt & jeweils 1 Punkt\\
    5. & 120 (90/60/30) Augen (bzw. ein Stich) gegen &\\
    &Absage unter 90 (60/30/schwarz) erreicht & jeweils 1 Punkt\\
    6. & hinzu kommen folgende Sonderpunkte: &\\
    & gegen die \kreuz -Damen gewonnen & 1 Punkt\\
    & Doppelkopf (Stich mit 40 oder mehr Augen) & 1 Punkt\\
    & \karo As (Fuchs) der Gegenpartei gefangen & 1 Punkt\\
    & \kreuz Bube (Karlchen) macht letzten Stich & 1 Punkt
  \end{tabular}
  \caption{Wertung des Spiels}
\end{table}
Falls beide Parteien ihr abgesagtes Ziel nicht erreichen, so hat keine Partei gewonnen und es werden nur die Punkte unter 3., 5. und 6. vergeben.

Bei einem Solo werden Sonderpunkte nicht gewertet. Diese Punktzahl wird für den Solospieler verdreifacht und ihm bei Gewinn gutgeschrieben, bei Niederlage abgezogen. Den drei Mitspielern der Gegenpartei wird die einfache Punktzahl mit umgekehrtem Vorzeichen angeschrieben.
\end{document}
