\documentclass[11pt,twocolumn,landscape]{scrartcl}

\usepackage{FreeDoko}

\svnInfo $Id$

% Keine Einrückung am Anfang eines Absatzes
\setlength{\parindent}{0pt}
% Zwischen Absätzen eine halbe Zeile Abstand
\KOMAoptions{parskip=half}
% Abstand zwischen den beiden Spalten
\setlength{\columnsep}{3em}

\renewcommand*\descriptionlabel[1]{\hspace\labelsep\normalfont\textit{#1}}

\begin{document}
\thispagestyle{empty}
\paragraph{Kartenreihenfolge:}~

\begin{tabular}{lr|lr}
Trumpf & Wert & Farbe & Wert\\
\hline
Herz Zehn & 10 &  &      \\
%          &    &  &      \\
Kreuz Dame & 3 &  Kreuz As & 11\\
Pik Dame & 3   &  Kreuz Zehn & 10\\
Herz Dame & 3  &  Kreuz König & 4\\
Karo Dame & 3  &Kreuz Neun  & 0\\
%          &    &           & \\
Kreuz Bube& 2  &   Pik As & 11\\
Pik Bube  & 2  &  Pik Zehn & 10\\
Herz Bube & 2  &  Pik König & 4\\
Karo Bube & 2  &  Pik Neun   & 0\\
%          &    &            & \\
Karo As   & 11 & Herz As    & 11\\
Karo Zehn & 10 &            & \\
Karo König & 4 & Herz König& 4\\
Karo Neun   & 0 & Herz Neun  &0\\
\end{tabular}

\paragraph{Sonderregeln}
\begin{compactdesc}
\item[Fuchs:] ein einzelnes Trumpf Karo As, das die Gegenpartei in einem Stich gewinnt, gibt einen Sonderpunkt
\item[Schweine:] beide Trumpf Asse auf einer Hand werden zu den höchsten Trümpfen (über den Herz Zehnen)
\item[Armut:] (ein Solo geht vor)
\begin{compactitem}
\item 3 Trümpfe oder weniger: Trümpfe verdeckt auf den Tisch legen, Partner ist der Spieler, der diese auf nimmt und entsprechend viele
beliebige Karten zurück gibt (dabei sagt er an wieviele Trümpfe er zurück gibt)
\item 5 Neunen oder mehr: es wird neu gegeben
\end{compactitem}
\item[Hochzeit:] Beide Kreuz Damen auf einer Hand: der Spieler darf ein Kriterium angeben, das entscheidet, wer mit ihm zusammen spielt
(erster fremder Stich, erster fremder Pik Stich, ...)
\end{compactdesc}

\pagebreak
\paragraph{Ansagen}~

\begin{tabular}{lll}
Bis Stichnr. & Ansage & Bedeutung\\\hline
2 & Re  & Kreuz Dame \\
  &     & keine 120 Punkte für Gegner\\
2 & Kontra & gegen die Kreuz Damen \\
  &        & keine 120 Punkte für Gegner\\
3 & keine 90 & keine 90 Punkte für Gegner\\
6 & keine 60 & keine 60 Punkte für Gegner\\
9 & keine 30 & keine 30 Punkte für Gegner\\
9 & Schwarz & keinen Stich für Gegner\\
\end{tabular}

Wenn der erste Stich 30 oder mehr Punkte hat, muß der Gewinner dieses Stiches
Re oder Kontra ansagen.

Re und Kontra verdoppeln den Wert des Spieles. Alle anderen Ansagen sind ein
Extrapunkt für die Gewinnerpartei.

In der Reihenfolge der Ansagen darf keine ausgelassen werden.

\paragraph{Soli}
\begin{compactdesc}
\item[Karosolo:] nichts ändert sich
\item[Farbsolo:] eine Farbe ersetzt die Karo-Karten als Trümpfe
\item[Damensolo:] nur Damen sind Trumpf
\item[Bubensolo:] nur Buben sind Trumpf
\item[Damen-Buben-Solo:] nur Damen und Buben sind Trumpf
\item[Köhler:] nur Könige, Damen und Buben sind Trumpf
\item[Fleischloser:] es gibt keine Trümpfe
\item[verdecktes Solo:] man hat eine Hochzeit und sagt diese nicht an
\end{compactdesc}

Jedes Solo zählt für den Solospieler dreifach.

Alle nicht mehr benötigten Trümpfe werden gegebenfalls eingereiht. Reihenfolge:
As, Zehn, König, Dame, Bube, Neun.
\end{document}
