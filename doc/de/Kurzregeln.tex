\documentclass[a4paper,11pt,twocolumn]{article}


\usepackage{german}

\parindent=0pt

\begin{document}
\thispagestyle{empty}
\textbf{Kartenreihenfolge:}\ \\
\smallskip\ \\
\begin{tabular}{lr|lr}
Trumpf & Wert & Farbe & Wert\\
\hline
Herz Zehn & 10 &  &      \\
          &    &  &      \\
Kreuz Dame & 3 &  Kreuz As & 11\\
Pik Dame & 3   &  Kreuz Zehn & 10\\
Herz Dame & 3  &  Kreuz K"onig & 4\\
Karo Dame & 3  &Kreuz Neun  & 0\\
          &    &           & \\
Kreuz Bube& 2  &   Pik As & 11\\
Pik Bube  & 2  &  Pik Zehn & 10\\
Herz Bube & 2  &  Pik K"onig & 4\\
Karo Bube & 2  &  Pik Neun   & 0\\
          &    &            & \\
Karo As   & 11 & Herz As    & 11\\
Karo Zehn & 10 &            & \\
Karo K"onig & 4 & Herz K"onig& 4\\
Karo Neun   & 0 & Herz Neun  &0\\
\end{tabular}
\ \\
\smallskip
\ \\
\textbf{Sonderregeln:}
\begin{itemize}
\item \textit{Fuchs:} ein einzelnes Trumpf Karo As, das die Gegenpartei in einem Stich gewinnt, gibt einen Sonderpunkt
\item \textit{Schweine:} beide Trumpf Asse auf einer Hand werden zu den h"ochsten Tr"umpfen ("uber den Herz Zehnen)
\item \textit{Armut:} (ein Solo geht vor)
\begin{itemize}
\item 3 Tr"umpfe oder weniger: Tr"umpfe verdeckt auf den Tisch legen, Partner ist der Spieler, der diese auf nimmt und entsprechend viele
beliebige Karten zur"uck gibt (dabei sagt er an wieviele Tr"umpfe er zur"uck gibt)
\item 5 Neunen oder mehr: es wird neu gegeben
\end{itemize}
\item \textit{Hochzeit:} Beide Kreuz Damen auf einer Hand: der Spieler darf ein Kriterium angeben, das entscheidet, wer mit ihm zusammen spielt
(erster fremder Stich, erster fremder Pik Stich, ...)
\end{itemize}
\newpage
\textbf{Ansagen:}\ \\
\smallskip\ \\
\begin{tabular}{lll}
Bis Stichnr. & Ansage & Bedeutung\\\hline
2 & Re  & Kreuz Dame \\
  &     & keine 120 Punkte f"ur Gegner\\
2 & Kontra & gegen die Kreuz Damen \\
  &        & keine 120 Punkte f"ur Gegner\\
3 & keine 90 & keine 90 Punkte f"ur Gegner\\
6 & keine 60 & keine 60 Punkte f"ur Gegner\\
9 & keine 30 & keine 30 Punkte f"ur Gegner\\
9 & Schwarz & keinen Stich f"ur Gegner\\
\end{tabular}
\medskip

Wenn der erste Stich 30 oder mehr Punkte hat, mu"s der Gewinner dieses Stiches
Re oder Kontra ansagen.\\
Re und Kontra verdoppeln den Wert des Spieles. Alle anderen Ansagen sind ein
Extrapunkt f"ur die Gewinnerpartei.\\
In der Reihenfolge der Ansagen darf keine ausgelassen werden.
\ \\
\smallskip
\ \\
\textbf{Soli:}
\begin{itemize}
\item \textit{Karosolo:} nichts "andert sich
\item \textit{Farbsolo:} eine Farbe ersetzt die Karo-Karten als Tr"umpfe
\item \textit{Damensolo:} nur Damen sind Trumpf
\item \textit{Bubensolo:} nur Buben sind Trumpf
\item \textit{Damen-Buben-Solo:} nur Damen und Buben sind Trumpf
\item \textit{K"ohler:} nur K"onige, Damen und Buben sind Trumpf
\item \textit{Fleischloser:} es gibt keine Tr"umpfe
\item \textit{verdecktes Solo:} man hat eine Hochzeit und sagt diese nicht an
\end{itemize}
Jedes Solo z"ahlt f"ur den Solospieler dreifach.\\
Alle nicht mehr ben"otigten Tr"umpfe werden gegebenfalls eingereiht. Reihenfolge:
As, Zehn, K"onig, Dame, Bube, Neun.
\end{document}
