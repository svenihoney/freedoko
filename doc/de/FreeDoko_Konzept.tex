\documentclass[12pt,a4paper]{article}

\usepackage[german]{babel}
\usepackage{tabularx}
%\usepackage{times}
\usepackage[pdftex]{hyperref}
\usepackage[utf8x]{inputenc}

\newcommand{\kreuz}{$\clubsuit$}
\newcommand{\pik}{$\spadesuit$}
\newcommand{\herz}{$\heartsuit$}
\newcommand{\karo}{$\diamondsuit$}


\begin{document}

\thispagestyle{empty}
\begin{center}
\linespread{2.0}
{\Huge
FreeDoko
\ \\
\vspace{2cm}
Konzept
\ \\
\vspace{2cm}
}
{\large
\ \\
\vspace{2cm}
Borg Enders\\
Diether Knof
\ \\
\vspace{2cm}
\ \\
}
\ \\
\vspace{2cm}
\ \\
September 2001
\end{center}
\newpage
\tableofcontents \newpage
\section{Spielidee}
Das Doppelkopfspiel ist ein Kartenspiel für vier oder mehr Personen.
Das einzelne Spiel wird von vier Spielern gespielt. Das Doppelkopfspiel ist
wahrscheinlich aus dem meist im süddeutschen Raum gespielten Schafkopf entstanden, das bereits seit 1895 ein Regelwerk
besaß. Es erhielt den Namen daher, daß alle Karten doppelt im Spiel vorhanden sind (Doppel-Schafkopf).
Doppelkopf wird im Gegensatz zu Schafkopf inzwischen in ganz Deutschland gespielt, überwiegend jedoch in Norddeutschland
und in der Rhein-Main-Gegend.\\
Doppelkopf ist ein Kartenspiel mit 48 Karten und mindestens 4 Spielern, das Skat nicht unähnlich ist. Hierbei wird mit zwei
normalen Skatblättern ohne Siebener und Achter gespielt. Es spielen dann immer die Besitzer der beiden Kreuz-Damen zusammen.
Hierdurch bilden sich während des Spielens immer wieder neue Paare. Es ist von vornherein nicht klar, wer mit wem eine Partei bildet.
Dies herauszufinden oder zu verheimlichen, macht einen wesentlichen Reiz des Doppelkopfspieles aus.\\
Bei Doppelkopf ist der Spielverlauf auch ähnlich wie bei Skat zu großen Teilen vom Können der einzelnen Spieler abhängig.
Dieses sorgt für ein spannendes und abwechslungsreiches Spielerlebnis.
\newpage
\section{Vorgesehener Funktionsumfang}
\subsection{Die unterstützten Regeln}
\begin{itemize}
\item Spiel mit oder ohne Neunen
\item Armut mit mehr als vier Neunen
\item Trumpfarmut bei weniger als vier Trümpfen
\item ein angesagter Vorbehalt ist verbindlich
\item Hochzeit (beide Kreuz-Damen auf einer Hand); Wahl des Mitspielers:
\begin{itemize}
\item Erster fremder Stich entscheidet
\item Erster fremder Herz Stich entscheidet
\item Erster fremder Pik Stich entscheidet
\item Erster fremder Kreuz Stich entscheidet
\item Erster fremder Farbstich entscheidet
\item Erster fremder Trumpfstich entscheidet
\item Festlegen des letzten Klärungsstiches
\end{itemize}
\item Schweine (beide Karo Asse auf einer Hand) sind die höchsten Trümpfe. Ansagezeitpunkt:
\begin{itemize}
\item Ansage vor dem Spiel
\item Ansage, wenn erstes Schwein gespielt wird
\item Wenn erstes As von der eigenen Partei gewonnen wurde, wird das zweite Karo As zum Schwein und angesagt
     (keine Hyperschweine)
\item Hyperschweine, beide Karo Neunen auf einer Hand, wenn Schweine angesagt wurden:
\begin{itemize}
\item Ansage vor dem Spiel
\item Ansage, wenn erste Neun gespielt wird
\end{itemize}
\end{itemize}
\item Ein Sonderpunkt für Karlchen: Kreuz Bube macht letzten Stich
\begin{itemize}
\item Zwei Sonderpunkte für Doppel-Karlchen: beide Kreuz Buben einer Partei im letzten Stich
\item Ein Sonderpunkt für Karlchen fangen: wenn Kreuz Bube im letzten Stich von Gegenpartei überstochen wird und
die Gegenpartei den Stich macht
\item Zwei Sonderpunkte für Doppel-Karlchen fangen: wenn beide Kreuz Buben im letzten Stich von Gegenpartei überstochen werden
\end{itemize}
\item Ein Sonderpunkt für Fuchs im letzten Stich: Letzter Stich wird mit Karo As (kein Schwein) gewonnen:
\begin{itemize}
\item Zwei Sonderpunkte für Doppel-Fuchs im letzten Stich: letzter Stich wird durch zwei Karo Asse einer Partei gewonnen
\end{itemize}
\item Ein Sonderpunkt für gefangenen Fuchs, wenn ein Karo As in einem Stich der Gegenpartei endet
\item Herz Zehnen Trumpf (oder normale Farbkarten). Bei Trumpf:
\begin{itemize}
\item Zweite Herz Zehn in einem Stich schlägt erste
\item Außer im letzten
\item Erste Herz Zehn schlägt zweite
\item Außer im letzten Stich
\item Ein Sonderpunkt für Dollenschlag, wenn eine Herz Zehn die andere schlägt
\end{itemize}
\item Genscher: Beide Karo Könige auf einer Hand, dann darf der Spiler mit dem Genscher sich beim Ausspielen
des ersten Karo Königs einen neuen Mitspieler aussuchen. Hierdurch werden alle Ansagen außer Re und Kontra nichtig.
Dieser ist nur bei Normalspielen oder geklärter Hochzeit erlaubt.
\item Soli: Ein Spieler spielt gegen die drei anderen:
\begin{itemize}
\item Buben-Solo: nur Buben sind Trumpf
\item Damen-Solo: nur Damen sind Trumpf
\item Damen-Buben-Solo: nur Damen und Buben sind Trumpf
\item Köhler: Nur Bilder sind Trumpf
\item Farbsolo: (Farb Schweine (optional),) (Herz Zehnen,) Damen, Buben und entsprechende Farbkarten sind Trumpf
\item Fleischloser: Es gibt keinen Trumpf
\end{itemize}
\item An- und Absagen (mit Angaben bis wann und viele Ansagen insgesamt erlaubt sind)
\begin{itemize}
\item Re: Ich habe eine Kreuz-Dame (bin Solospieler)
\item Kontra: Ich habe keine Kreuz-Dame (bin kein Solospieler)
\item Keine 90: die Gegenpartei schafft keine 90 Punkte, Ansage der eigenen Partei mußte vorher erfolgen
\item Keine 60: die Gegenpartei schafft keine 60 Punkte, keine 90 mußte vorher angesagt werden
\item Keine 30: die Gegenpartei schafft keine 30 Punkte, keine 60 mußte vorher angesagt werden
\item Schwarz: die Gegenaprtei macht keinen Stich, keine 30 mußte vorher angesagt werden
\item Absagen dürfen übersprungen werden, wenn vorher nötige Absagen noch erlaubt sind. Diese gelten
dann als abgesagt
\item Auf jede Absage darf ein Re oder Kontra ggf. erwidert werden
\end{itemize}
\item Eine vorgegeben Anzahl (abhängig vom Schwierigketisgrad) von bereits gespielten Stichen
darf betrachtet werden
\item Frei wählbare Anzahl und Art der Pflichtsoli
\item Frei wählbare Zählart, ob Plus-Minus, Plus oder Minus
\end{itemize}
\subsection{Spielumfang}
\begin{itemize}
\item Einzelspieler Spiel gegen KIs unterschiedlicher Schwierigkeitsgrade
\item Multiplayer-Spiel
\item Verschiedene Spielvarianten
\begin{itemize}
\item Vorgegebene Anzahl von Runden
\item Spiel bis zum erreichen einer bestimmten Punktzahl
\item Spielen einer bestimmten Anzahl von Punkten (Punktetopf)
\item Endlosspiel
\item Spiel auf Zeit (Tunier)
\item Regeln durch Anwender weitestgehend konfigurierbar
\end{itemize}
\item Speichern und Laden von Doppelkopfpartien
\item Sortierung der Karten
\begin{itemize}
\item aufwärts/abwärts
\item Trumpf links/rechts
\item beliebige Farbreihenfolge sollte angebbar sein: Herz Zehnen, Karo, Kreuz, Pik, Herz, Schweine, Buben, Damen
\end{itemize}
\end{itemize}
\subsection{Technischer Umfang}
\begin{itemize}
\item Rechneranforderungen ca. 300MHz und 64MB.
\item Unterstützte Auflösungen $640\times 480$, $800\times 600$, $1024\times 768$, $1280\times 1024$
\item Eine Programmversion für Linux und eine für Windows 95 und höher
\item Als graphische Oberfläche wird GTK+ verwendet
\item Alle Funktionen über Tastatur und Maus bedienbar
\item Konfigurationsdatei
\item Kommandozeilenargumente
\item Programmiersprache: C++ (Compiler: Linux: g++; Windows: g++)
\item eine zusätzliche textbasierte Version für Linux und Windows
\item Multiplayerfähig via TCP/IP
\begin{itemize}
\item peer-to-peer-Netz: Hostaufagben können einfach weiter gereicht werden beim ausscheiden des Hosts
\end{itemize}
\item Soundausgaben (linux: sox (Systemaufruf); Windows: PlaySound aus der Multimedia-Bibliothek winmm.lib)
\item Mehrsprachig (deutsch, englisch, ...)
\item Wird unter die GPL gestellt
\end{itemize}
\subsection{Dokumentation}
\begin{itemize}
\item Anleitungen:
\begin{itemize}
\item Die DDV-Kurzanleitung als Crashkurs für ganz eilige.
\item Die FreeDoko-Regeln als detailierte Einführung in das Spiel.
\item Die Tunier-Regeln für alle die es ganz genau wissen wollen.
\end{itemize}
\end{itemize}
\subsection{Spielerprofil}
Der Spieler gibt Startlevel an und kann seinen Level beliebig up/downgraden.\\
Nach x gewonnen Spielen (Punkte der eigenen Partei \textgreater 120 und eigene Punkte \textgreater 60), nach y gehaltenen Absagen
und nach z gewonnen Soli findet ein Upgrade des Spielerlevels statt.\\
Durch ein freiwilliges up/downgraden werden die ermittelten Gesamtwerte nicht beeinflußt.
Ein Downgrade bringt den Spieler zur Mitte des letzten Levels zurück, ein Upgrade zum Anfang des nächsten Levels.\\
Die KI wird dann entsprechend dem Spielerlevel gewählt und andere Spieler können sehen, wie gut ihr Gegener ist.
Jeder Spieler darf auch eine Ersatz KI, die sein Spiel beim Ausfall seines Rechners Übernimmt, entsprechend seinem
Level festlegen.
\subsection{Schwierigkeitsgrade}
\subsubsection{Anfänger}
\begin{itemize}
\item Alle Karten offen
\item Alle Stiche betrachtbar
\item Re/Kontra anzeigen
\item Schweine anzeigen
\item Automatische Punktezählung
\item Kartenvorschlag
\item KI zählt Herz Zehnen und Schweine mit
\item KI berücksichtigt keine weiteren gespielten Karten
\end{itemize}
\subsubsection{Anfänger pro}
\begin{itemize}
\item Karten verdeckt
\item Alle Stiche betrachtbar
\item Re/Kontra anzeigen
\item Schweine anzeigen
\item Automatische Punktezählung
\item Kartenvorschlag
\item KI zählt Herz Zehnen und Schwein mit
\item Die KI zählt gestochene Farben mit
\end{itemize}
\subsection{Normal}
\begin{itemize}
\item Nur letzten Stich noch betrachten
\item Re/Kontra anzeigen
\item Schweine Anzeigen
\item Automatische Punktezählung
\item Kartenvorschlag
\item KI zählt gestochene Farben mit
\item KI berücksichtig letzten Stich
\item KI zählt Herz Zehnen und Schweine mit
\end{itemize}
\subsection{Normal pro}
\begin{itemize}
\item Keine Punkte zählen
\item Nur letzten Stich noch betrachten
\item Re/Kontra anzeigen
\item Schweine anzeigen
\item Kartenvorschlag
\item KI zählt gestochene Farben mit
\item KI berücksichtigt letzte 3 Stiche
\item KI zählt Herz Zehnen und Schweine mit
\end{itemize}
\subsection{Profi}
\begin{itemize}
\item Keine Punkte zählen
\item Nur letzten Stich noch betrachten
\item Schweine anzeigen
\item KI zählt gestochene Farben mit
\item KI berücksichtigt letzte 6 Stiche
\item KI zählt Herz Zehnen und Schweine mit
\end{itemize}
\subsection{Profi pro}
\begin{itemize}
\item Keine Punkte zählen
\item Nur letzten Stich noch betrachten
\item KI zählt gestochene Farben mit
\item KI berücksichtigt alle Stiche
\item KI zählt Herz Zehnen und Schweine mit
\end{itemize}
\newpage
\section{GUI}
\subsection{Entwurf}
Fähigkeiten der GUI:
\begin{itemize}
\item Stiche betrachten
\item Punkte zählen
\item Einstellungen
\begin{itemize}
\item Sortierung der Karten
\item Kartenart wählen (auf jedenfall die Rückseite)
\end{itemize}
\item Auswahl von An- und Absagen
\end{itemize}
\subsection{Die Grafik}
\begin{itemize}
\item Blick von oben auf einne Tisch mit Schreibblock (vergrößerbar), Karten an den vier Tischseiten und aktueller Stich
in der Mitte.
\item nach Möglichkeit menschliche Hände und ähnliches vermeiden, da gute Grafik hierfür zu aufwendig in der Erstellung sein würde
\item Animationen:
\begin{itemize}
\item die Karten werden in die Mitte geschoben
\item dann wird der Stich umgedreht
\item zu dem Spieler geschoben, der ihn bekommt
\item vor dem dann ein Kartenstapel anwächst
\end{itemize}
\end{itemize}
\newpage
\section{Programmablauf}
\subsection{Server-Klient-Konzept}
Idealerweise sollte der Server nicht als separates Programm gestartet werden, sondern kann vom Spieler eingestellt werden
(beim Programmstart wird er automatisch gestartet, es kann aber auch ein anderer eingestellt werden).
\begin{itemize}
\item Vorbereitung:
\begin{itemize}
\item Einstellungen beim Server machen (z.Bsp.: Spielregeln, Art der Klients, ...)
\item Einstellungen bei den Klients machen (z.Bsp.: Namen, Schwierigkeitsgrad, ...)
\item Auf alle Klients warten
\item Server abgleichen
\end{itemize}
\item Partie starten:
\begin{itemize}
\item Spiel starten
\begin{itemize}
\item Karten mischen
\item Karten austeilen
\end{itemize}
\item Ansagen abwarten
\item Stiche machen
\begin{itemize}
\item Züge durchführen
\begin{itemize}
\item Ausspieler bekannt geben
\item von Ausspieler Karte verlangen
\item ausgespielte Karte den Klients zeigen
\end{itemize}
\item Stich einsammeln
\item Gewinner bekannt geben
\end{itemize}
\item Spielende
\begin{itemize}
\item Punkte der Parteien zählen
\item Sonderpunkte vergeben
\item Punkte in die Tabelle eintragen
\item Punktestand angeben
\end{itemize}
\end{itemize}
\item Punktestand und Gewinner angeben
\end{itemize}
\subsubsection{Server}
\begin{itemize}
\item Mischen der Karten
\item Verteilen der Karten
\item Festlegen der Spielregeln
\item Übermitteln der Spielregeln an die Klients
\item Verwalten der Punkte (Aufschreiben)
\item Kommunikation mit den Klients
\item Einstellung der Art der Klients (Mensch oder KI)
\end{itemize}
\subsubsection{Klient}
\begin{itemize}
\item Name einstellbar
\item Einsicht letzter Stich
\item Bekommt vom Server die Regeln
\item Überprüfen der Spielregeln
\item Bekommt vom Server die gespielten Karten
\item Spielt Karte aus
\end{itemize}
\subsection{peer-to-peer}
Jeder Rechner kennt alle am Spiel beteiligten Rechner.
Auf jedem Rechner wird festgehalten, welcher Rechner der aktuelle Server ist.\\
Wenn ein Rechner Server ist, ermittelt dieser die nächste gespielte Karte entweder von Spieler auf einem der Klients
oder von lokaler KI.
Lokal ermittelte Karten werden von Klients und Server an alle anderen beteiligten Klients und den ggf. den Server geschickt.
Nach Ablauf des Stiches wertet der Server den Stich aus und gibt die Ergebnisse an die Klients weiter.\\
Serverausfall: Wenn ein Klient seit 0.3 sec nichts vom Server gehört hat, fragt dieser beim Server nach, ob er noch aktiv ist.
Falls keine Antwort erfolgt, sendet der Klient den anderen Rechnern das Vorliegen eines Serverausfalls.
Jeder Rechner bekommt beim Anmelden beim Server noch eine Numer und jetzt wird der Rechner mit der niedrigsten noch existierenden
Nummer zum Server. Der ausgefallene Server wird dabei ggf. durch eine zusätzliche KI ersetzt.
\newpage
\section{KI}
\subsection{Funktionsumfang}
\begin{itemize}
\item Schwierigkeitsstufe einstellbar
\begin{itemize}
\item Gedächtnis (wie viele Stiche werden sich gemerkt, auf welche Karten wird geachtet, ...)
\item Vorraussicht, wie viele Stiche in die Zukunf wird der Spielbveraluf berechnet
\item Mitzählen von abgestochen Farben, Herz Zehnen, Schweinen
\end{itemize}
\item Verhalten einstellbar
\begin{itemize}
\item gierig (nimmt jeden Stich, den er kriegen kann)
\item vertrauensvoll (wenn ihn jemand eine Karo 10 gibt, dann gibt er ihm auch Punkte)
\end{itemize}
\end{itemize}
\subsection{Heuristiken}
Was ist eine Heuristik? Einfach gesagt eine Regel, die meistens aber nicht immer gilt. Im folgenden
werden einige Heuristiken für Doppelkopf mit absteigender Anwendungsbedeutung aufgeführt, das heißt,
es ist sinnvoller die erste Regel zuerst anzuwenden, als die zweite:
\begin{enumerate}
\item Komme mit einem Farb-As beraus, wenn noch entsprechend viele Karten der Farbe im Spiel sind (und keiner
diese Farbe bereits gestochen hat). Komme dabei mit dem As heraus, wo noch am meisten Karten ausliegen.
\item Falls kein As als Ausspieler, dann kleinen Trumpf nehmen.
\item Als letzte Karte eines Farbstiches, dessen Farbe man nicht hat, spiel ein Karo As, eine Karo Zehn,
einen Karo König, einen Buben, eine Karo Neun oder eine Dame,
\begin{enumerate}
\item wenn in diesem Stich nicht bereits eine Trumpfkarte ist (bei Gierig überstechen).
\item wenn nicht bereits jemand von der eigenen Partei den Stich bekommt (vertrauensvoll: pfundet).
\end{enumerate}
\item Wenn die eigene Partei den Stich gewinnt, pfunde durch Spielen eines Karo Asses, einer Karo Zehn, einer Farb Zehn, eines Farb
Asses oder Königs als letzte Karte, sofern erlaubt.
\item Immer so niedrig wie möglich bedienen (außer der Partenr steht schon fest)
\item Wenn man den mit einem As oder einer Zehn den Stich nicht gewinnt, versucht man doch lieber einen König oder eine
Neun der entsprechenden Farbe zu spielen.
\item Wenn mit einem Trumpf dieser Stich nicht gewonnen wird, sollte Fehl oder ein niedriger Trumpf gespielt werden, wenn erlaubt. Die Trumpfreihefolge hierbei ist zuerst
Karo Neun, dann Karo König, Bube, Karo Zehn, Karo As, Dame.
\item Spiel kein As, das bereits im Stich liegt. Es sei denn der Partner bekommt den Stich.
\item Spiel keine Dame die bereits im Stich leigt (Damen sind in der Regel zu wertvoll, als sie zu opfern, wenn man es vermeiden kann).
\item auf Karlchen spielen wenn möglich
\item im Wechsel hohe niedrige Trümpfe spielen
\end{enumerate}
\subsubsection{Armut}
Eine Armut kann aufgenommen werden, wenn der Spieler mindestens ein paar Damen hat und durch die Armut nur Trümpfe
oder max. 2 Fehlfarben hat. Wenn der Spieler in Vorhand sitzt, spielt er einen niedrigen Trumpf, ansonsten versucht er
seine Fehlfarben loszuwerden bzw. sticht die Farben ab. Die Trümpfe sollten meist im Wechsel gespielt werden, das heißt, mal einen
niedrigen mal einen hohen. Wenn der Spieler viele hohe hat, kann er auch von oben ziehen.
\subsubsection{Solo}
\begin{itemize}
\item Farbsolo
\begin{itemize}
\item der Spieler sollte viele Trümpfe haben bzw. ein gutes Beiblatt, das heißt Asse.
\item Entweder sollte dann ein As oder ein niedriger Trumpf gespielt werden
\end{itemize}
\item Buben / Damen /gemischtes
\begin{itemize}
\item der Spieler sollte ein einigermaßen gutes Beiblatt haben
\item Ansonsten wie das Farbsolo spielbar
\end{itemize}
\item Köhler
\begin{itemize}
\item der Spieler sollte mindestesn 10 Trümpfe haben
\item als Fehlfarben am besten nur Asse
\item Er sollte einige Könige und Damen haben
\item dann im Wechsel hohe und niedrige Trümpfe spielen
\end{itemize}
\item Fleischloser
\begin{itemize}
\item Der Spieler sollte eine lange Farbe oder viele Asse haben
\item von oben runter spielen, oder wenn man alle oberen Karten selber hat, auch mal einen König vorspielen
\end{itemize}
\end{itemize}
\newpage
\section{Weiterführende Ideen}
\begin{itemize}
\item im Multiplayerspiel Zuschauer erlauben
\item im Multiplayerspiel ein öffentliches Chat-Fenster
\begin{itemize}
\item ermöglichen von Strafpunkte durch regelwidrigen Gebrauch des Chat-Fensters (Absprachen, Mitzählen, ...)
\end{itemize}
\item Online Tuniere
\item Doppelkopf mit 5 Spielern
\item Spielerprofil um frei einbindbare Spielergrafik erweitern
\item Frei konfigurierbarer Schwierigkeitsgrad
\item Doppelkopf mit 8 Spielern
\item (noch eine schwerere Spielstufe, bei der die KI die Kartenverteilung kennt)
\item Ai freikonfogurierbar (Heuristiken Liste mit Rangfogle, der
Heuristiken SPielreiehnfolge)
\end{itemize}
\end{document}
