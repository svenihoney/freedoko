\documentclass[12pt]{scrartcl}

\usepackage{FreeDoko}

\begin{document}

\thispagestyle{empty}
\begin{titlepage}
  \begin{center}
    \Huge \textbf{Turnierregeln}
  \end{center}
  \tableofcontents
\end{titlepage}

Dies ist die offizielle Fassung der Turnierspielregeln des
Deutschen Doppelkopf Verbandes e.V. (DDV).


\section{Definition der im folgenden genannten Personen}
\begin{description}
  \item[Teilnehmer:] Alle bei einem Doppelkopf"|turnier mitspielenden Personen.
  \item[Mitspieler:] Alle Personen am gleichen Tisch.
  \item[Spieler:] Die im jeweiligen Spiel mitspielenden vier Personen.
  \item[Gegenpartei:] Die jeweils andere Partei. Für einen Spieler der Kontra-Partei ist die Re-Partei die Gegenpartei und für einen Spieler der Re-Partei ist die Kontra-Partei die Gegenpartei.
  \item[Gegenspieler:] Ein Spieler der Gegenpartei.
  \item[Partner:] Ein Spieler der eigenen Partei.
  \item[Zuschauer:] Personen, die sich an einem Tisch aufhalten, an dem sie nicht selbst Mitspieler sind.
\end{description}


\section{Allgemeines}
Mit der Gründung des Deutschen Doppelkopfverbandes (DDV) wurden Bestrebungen in Gang gesetzt die Regeln zu vereinheitlichen.  Gerade das Kartenspiel Doppelkopf wurde zuvor regional nach sehr unterschiedlichen Regeln und Varianten gespielt und wird es in vielen privaten Spielrunden noch heute. Es ist jedoch für ein Kartenspiel nicht günstig, wenn man sich vor dem eigentlichen Spiel mit neuen Mitspielern zuerst über die Regeln unterhalten muß. Daher würde sich der DDV freuen, wenn die Turnierspielregeln in vielen privaten Runden bekannt gemacht würden. Falls es Fragen zu den Bestandteilen der Regeln geben sollte, können diese gerne an die Regelkommission des DDV gestellt weren. Die Anschriften sind über die Homepage \href{http://www.doko-verband.de}{www.doko-verband.de} des DDV zu finden.

\subsection{Das Doppelkopfblatt}
Das Doppelkopf-Kartenspiel besteht aus 48 Spielkarten in vier
Farben zu je 12 Karten. Die Farben heißen "`Kreuz"', "`Pik"',
"`Herz"' und "`Karo"'. jede Karte ist doppelt im Spiel. Hieraus
ergibt sich folgende Darstellung der Karten:
\\
\begin{tabular}{lcccccccccccc}
  &As     &As     &10     &10     &König &König &Dame   &Dame   &Bube   &Bube   &9   &9\\
  Kreuz &\Kreuz &\Kreuz &\Kreuz &\Kreuz &\Kreuz &\Kreuz &\Kreuz &\Kreuz &\Kreuz &\Kreuz &\Kreuz &\Kreuz \\
  Pik   &\Pik   &\Pik   &\Pik   &\Pik   &\Pik   &\Pik   &\Pik   &\Pik   &\Pik   &\Pik   &\Pik   &\Pik \\
  Herz  &\Herz  &\Herz  &\Herz  &\Herz  &\Herz  &\Herz  &\Herz  &\Herz  &\Herz  &\Herz  &\Herz  &\Herz  \\
  Karo  &\Karo  &\Karo  &\Karo  &\Karo  &\Karo  &\Karo  &\Karo  &\Karo  &\Karo  &\Karo  &\Karo  &\Karo  \\
  Augen & 11    &  11   & 10    &  10   &  4    & 4     &  3    & 3     &  2    &  2    &  0    &  0   \\
\end{tabular}\\
In jeder Farbe sind daher 60 Augen vorhanden, so daß das Doppelkopfblatt insgesamt 240 Augen enthält.

\subsection{Rangfolge der Karten}
Jede Karte gehört entweder zu einer der gleichwertigen
"`Fehlfarben"' oder zu den Trumpfkarten. Jede Trumpfkarte ist
ranghöher als irgendeine Fehlkarte.

Die Rangfolge hängt vom Typ des Spieles ab, der vor Beginn des
Spieles in der Vorbehaltsfrage ermittelt wird. Es gibt folgende
zwei Spieltypen: Normalspiel und Solo. Beim Solo werden
desweiteren vier Varianten unterschieden: Farbsolo, Damensolo,
Bubensolo, Fleischloser (Assesolo).

\subsubsection{Rangfolge beim Normalspiel:}
Die absteigende Rangfolge der 13 jeweils zweifach im Spiel
vorhandenen Trumpfkarten ist: \Herz\ Zehn, \Kreuz\ Dame, \Pik\ Dame,
\Herz\ Dame, \Karo\ Dame, \Kreuz\ Bube, \Pik\ Bube, \Herz\ Bube, \Karo
Bube, \Karo\ As, \Karo\ Zehn, \Karo\ König, \Karo\ Neun. In den drei
Fehlfarben ist die absteigende Rangfolge der ebenfalls doppelt
vorhanden Karten: In Kreuz und Pik: As, Zehn, König, Neun. In
Herz: As, König, 9. Beim Normalspiel gibt es damit 26
Trumpfkarten und 22 Fehlkarten.

\subsubsection{Rangfolge beim Farbsolo:}
Beim Farbsolo werden 4 Varianten unterschieden, da jede der 4
Farben als Trumpffarbe gewählt werden kann. Gegenüber dem
Normalspiel ersetzen die Trumpfkarten (As, Zehn (außer \Herz
-Solo), König, Neun) der gewählten Trumpffarbe die Trumpfkarten
der Farbe Karo. Damit ergibt sich folgende absteigende Rangfolge
der jeweils zweifach im Spiel vorhandenen Trumpfkarten: Wie beim
Normalspiel: \Herz\ Zehn, \Kreuz\ Dame, \Pik\ Dame, \Herz\ Dame, \Karo
Dame, \Kreuz\ Bube, \Pik\ Bube, \Herz\ Bube, \Karo\ Bube. Zusätzlich
aus der gewählten Trumpffarbe: As, Zehn (außer \Herz\ Zehn),
König, Neun. In den drei Fehlfarben ist die absteigende
Rangfolge der ebenfalls doppelt vorhanden Karten: As, Zehn (außer
\Herz\ Zehn), König, Neun. Beim Farbsolo in Kreuz, Pik und Karo
gibt es jeweils 26 Trumpfkarten und 22 Fehlkarten, und beim
Farbsolo in Herz
gibt es 24 Trumpfkarten und 24 Fehlkarten.\\
Rangfolge beim Damensolo und beim Bubensolo:\\
Die absteigende Rangfolge der 4 jeweils zweifach im Spiel
vorhanden Trumpfkarten ist: Beim Damensolo: \Kreuz\ Dame, \Pik
Dame, \Herz\ Dame, \Karo\ Dame. Beim Bubensolo: \Kreuz\ Bube, \Pik
Bube, \Herz\ Bube, \Karo\ Bube. In den Vier Fehlfarben ist die
absteigende Rangfolge der ebenfalls doppelt vorhandenen Karten:
Beim Damensolo: As, Zehn, König, Bube, Neun. Beim Bubensolo: As,
Zehn, König, Dame, Neun. Beim Damensolo und beim Bubensolo gibt
es jeweils 8 Trumpfkarten und 40 Fehlkarten.

\subsubsection{Rangfolge beim Fleischlosen (Assesolo):}
Beim Fleischlosen (Assesolo) gibt es keine Trumpfakrten und 48
Fehlkarten. In den vier Fehlfarben ist die absteigende Rangfolge
der doppelt vorhanden Karten jeweils: As, Zehn, König, Dame,
Bube, Neun.

\subsection{Die Parteienbildung}
Doppelkopf ist ein Parteienspiel. Die vier Spieler bilden zwei
Parteien: die Re-Partei und die Kontra-Partei. Es werden
grundsätzlich zwei Spieltypen unterschieden das Normalspiel und
das Solo.

\subsubsection{Das Normalspiel}
Falls kein Sollo gespielt wird, bilden die beiden Spieler mit den
Kreuz-Damen die Re-Partei, die beiden anderen Spieler die
Kontra-Partei. Wenn ein Spieler nach dem Austeilen beide
Kreuz-Damen auf der Hand hat, tritt der Sonderfall einer
"`Hochzeit"' ein. \textit{Beachte:} Es ist von vornherein nicht
klar, wer mit wem eine Partei bildet. Dies herauszufinden oder zu
verheimlichen, macht einen wesentlichen Reiz des
Doppelkopfspieles aus. Daher ist es unzulässig, durch irgendeine
nicht in den Spielregeln vorgesehene Aktion anzuzeigen, ob man
"`mit"' oder "`ohne"' Kreuz-Dame spielt.
\\
Eine Spieltaktik ist ein Spielzug, mit dem indirekt, das heißt,
ohne An- oder Absage, die vermeintliche Parteizugehörigkeit oder
eine andere Information mit hoher Wahrscheinlichkeit preisgegeben
wird. Das Anwenden einer Spieltaktik stellt keinen Kartenverrat
dar, da sie auch als Täuschung eingesetzt werden darf.

\subsubsection{Das Solo}
Ein Spieler kann versuchen, ein Spiel allein gegen die übrigen
drei Spieler zu gewinnen. Beim Solo bildet der Solospieler die
Re-Partei, die drei anderen Spieler die Kontra-Partei.

\subsection{Spielziel}
Jede Partei versucht mehr "`Augen"' als die Gegenpartei zu
erspielen. Weitere Ziele sind die Erspielung von Sonderpunkten
sowie für die stärkere Partei, den Punktwert des Spieles durch
An- bzw. Absagen soweit wie möglich zu erhöhen bzw. für die
schwächere Partei, den von der anderene Partei angesagten
Augenwert zu erreichen.


\section{Spielvorbereitung}

\subsection{Listenführer}
Der Mitspieler auf Position 4 hat die Liste zu führen. nach
Vereinbarung am Tisch kann auch ein anderer Mitspieler die
Listenführung übernehmen, die Sitzreihenfolge bleibt dadurch
jedoch unverändert.

\subsection{Bestimmung der Plätze}
Die Reihenfolge der Mitspieler wird ausgelost. Falls diese
Reihenfolge nicht auf dem Spieltisch gekennzeichnet ist und keine
Einigkeit mit den Mitspielern erzielt werden kann, wählt der
Listenführer seinen Platz. Die anderen Mitspieler schließen
sich im Uhrzeigersinn an.

\subsection{Geben der Karten}
Mit dem Geben der Karten zum ersten Spiel beginnt der Mitspieler
auf Position 1. Danach gibt der Spieler auf Position 2 usw.
Lediglich nach einem Pflichtsolo (jedoch nicht dem "`Vorführen"')
werden die Karten vom selben Geber noch einmal ausgeteilt. Wenn
der Geber vorübergehend abwesend ist, darf der links neben ihm
sitzende Mitspieler die Karten für das nächste Spiel mischen
und austeilen -- vorausgesetzt der eigentliche Geber hat sich das
Geben nicht ausdrücklich vorbehalten. Der Kartengeber hat die
Karten gründlich zu mischen, sie vom rechten Nachbarn einmal
abheben zu lassen, den dabei liegenbleibenen Teil auf den
abgehobenen zu legen und danach die Karten zu verteilen. Falls
die Karten beim Mischen vom Kartengeber "`gestochen"' oder
"`geblättert"' werden, ist vor dem Abheben noch einmal
durchzumischen. Abheben ist Pflicht! Es hat so zu erfolgen, daß
mindestsn drei Karten abgehoben werden bzw. liegen bleiben. Ist
der Abheber vorübergehend abwesend, darf der rechts neben ihm
sitzende Mitspieler abheben -- vorausgesetztz, der eigentliche
Abheber hat sich das Abheben nicht ausdrücklich vorher
vorbehalten. Es müssen beim linken Nachbarn des richtigen Gebers
beginnend, jedem Spieler viermal jeweils drei Karten gegeben
werden, so daß danach jeder Spieler 12 Karten in der Hand hält.
Die Karten sind in einer Weise zu geben, daß ihre Vorderseiten
keinem Spieler sichtbar werden. Falls beim Geben durch den
Kartengeber allein- oder mitverschuldet eine Karte aufgeworfen
wird, muß erneut gemischt, abgehoben und gegeben werden. Jeder
Spieler ist verpflichtet, darauf zu achten, daß ihm die richtige
Kartenanzahl (also 12) ausgeteilt wurde. Einsprüche gegen
jegliche sofort offensichtliche Unkorrektheit beim Mischen,
Abheben, sowie die Art und Weise der Kartenverteilung kann ein
Spieler nur geltend machen, solange er selbst noch keine seiner
Karten aufgenommen hat. Sofort offensichtliche Regelverstöße,
die erst während der Kartenaufnahme erkannt weren können (zum
Beispiel Kartenverteilung 3-3-4-2 oder das versehentliche
Aufdecken einer Karte am Ende des Gebens), dürfen auch noch bei
ihrer Erkennung reklamiert werden. Ein Verstoß gegen falscher
Geber ist in diesem Sinne kein sofort offensichtlicher
Regelverstoß.


\section{Spielfindung}

\subsection{Die Vorbehaltsfrage}
Bevor zum ersten Stich aufgespielt bzw. eine Ansage getroffen
werden darf, muß in der Vorbehaltsabfrage der Spieltyp ermittelt
werden. Beginnend beim linken Nachbarn des Kartengebers wird in
ununterbrochener Reihenfolge nach Vorbehalten abgefragt. Wenn ein
Spieler ein Pflichtsolo, Lustsolo oder eine Hochzeit anmelden
möchte, sagt er laut und deutlich: "`Vorbehalt"'. Die Anmeldung
eines Vorbehaltes kann an Position 1, 2 und 3 nur zurückgenommen
werden, wenn der nächste Spieler sich noch nicht geäußert hat.
Der Spieler an Position 4 kann seinen Vorbehalt nur zurücknehmen,
wenn weder eine Karte aufgespielt wurde, noch bereits eine Ansage
getätigt wurde, noch die Vorbehaltsfrage bei mehreren
Vorbehalten bereits fortgesetzt wurde. Andernfalls muß eventuell
ungewollt ein Solo gespielt werden. Ein Spieler der keinen
"`Vorbehalt"' anmelden möchte, sagt laut und deutlich:
"`gesund"'. Falls sich alle Spieler "`gesund"' melden wird ein
Normalspiel durchgeführt. hat nur ein Spieler "`Vorbehalt"'
angemeldet, muß er diesen "`taufen"', das heißt, er muß bekannt
geben, ob er ein Solo (Pflichtsolo oder Lustsolo) oder eine
Hochzeit spielen will. Im Falle eines Pflichtsolos erhält dieser
Spieler die Aufspielpflicht zum ersten Stich. Haben mehrer Spieler
im selben Spiel "`Vorbehalt"' angemeldet, gilt die Reihenfolge:
\begin{enumerate}
  \item Pflichstolo,
  \item Lustsolo,
  \item Hochzeit.
\end{enumerate}
Um festzustellen, wer bei mehreren Vorbehalten den
höchstrangigen hat, wird als erstes reihum gefragt, ob einer der
betroffene Spieler ein Pflichtsolo spielen will. Wenn dies alle
verneinen, wird nach dem Lustsolo gefragt. Haben mehrere Spieler
"`Vorbehalt"' angemeldet, kann der Vorbehalt "`Hochzeit"' niemals
zum Zuge kommen. Denn bei mehreren gleichrangigen "`Vorbehalten"'
hat der "`weiter vorn sitzende"' Spieler den Vorang und die
Abfrage endet, sobald ein Spieler den gefragten Vorbehalt bejaht.
Alle weiteren Vorbehalte sind ohne Bedeutung und dürfen auch
nicht mehr getauft werden. Ist es für einen Spieler sicher, daß
sein Vorbehalt der höchstrangige ist, kann er ihn sofort ansagen
("`taufen"'), auch wenn er nicht an der Reihe ist. Es darf danach
sofort aufgespielt werden, ohne daß die Vorbehaltsfrage zu Ende
geführt wird. Die anderen Spieler dürfen dann selbst weder
einen Vorbehalt anmelden noch diesen taufen. Bei einer
"`Vorführung"' steht fest, welcher Spieler ein Pflichtsolo
spielt. Es wird daher keine Vorbehaltsabfrage durchgeführt. Der
Solospieler gibt lediglich bekannt, welches Solo er spielen wird.

\subsection{Pflichtsolo}
Jede Soloart kann als Pflichtsolo gespielt werden. Der
Solospieler hat hierbei die Aufspielpflicht zum ersten Stich.
Jeder Spieler muß ein Pflichtsolo innerhalb der Spielrunde
absolvieren. Ist die Anzahl nicht gespielter Pflichtsoli gleich
der Anzahl der noch zu absolvierenden Spiele, muß derjenige
Spieler, dem ein Pflichtsolo fehlt und der als Nächster links vom
Kartengeber sitzt, ein Pflichtsolo spielen, das heißt, er wird
"`vorgeführt"'. Es ist nicht möglich, ihm dieses Spiel
abzunehmen. Kartengeber ist derjenige Spieler, der das nächste
Spiel geben müßte, auch wenn er selbst vorgeführt wird. Das
Pflichtsolo wird in der Spielliste in der hierfür vorgesehenen
Rubrik notiert. nach einem Pflichtsolo werden die Karten vom
selben Geber, der das Pflichtsolo gegeben hat, noch einmal
gegeben. Wenn das Pflichtsolo jedoch als vorgeführtes Solo
gespielt wurde, gibt der nächste Mitspieler.

\subsection{Lustsolo}
Ein Lustsolo darf erst dann gespielt werden, wenn das eigene
Pflichtsolo bereits gespielt wurde. Als Lustsolo kann jede
Soloart gespielt werden, ohne Rücksicht auf die Art der zuvor
gespielten Soli. Beim Lustsolo spielt der linke Nachbar des
Kartengebers zum ersten Stich auf. Das Lustsolo wird in der
Spielliste in der Rubrik der Normalspiele notiert. Das nächste
Spiel wird deshalb von einem anderen Geber ausgeteilt.

\subsection{Hochzeit}
Wenn ein Spieler beide Kreuz-Damen in der Hand hat, kann er sich,
falls er kein Solo melden möchte, zwischen zwei Spieltypen
entscheiden. Er kann dann zwischen einer angemeldeten Hochzeit
und einer "`Stillen Hochzeit"' (Farbsolo in Karo) wählen. Falls
der Spieler mit den beiden Kreuz-Damen kein Solo spielen möchte,
meldet er in der Vorbehaltsfrage "`Vorbehalt"' an, ohne seinen
Vorbehalt gleich zu taufen. Wenn kein anderer Spieler einen
Vorbehalt angemeldet hat, tauft er seinen Vorbehalt, indem er laut
und deutlich "`Hochzeit"' sagt.

Die angemeldete Hochzeit:\\
Eine angemeldete Hochzeit kann unterschiedliche Spielverläufe haben:
\begin{enumerate}
  \item
    Es spielt derjenige als "`Re-Mann"' mit dem Hochzeiter, der
    innerhalb der ersten drei Stiche dieses Spieles den ersten Stich
    (der sogenannte Klärungsstich) -- abgesehen von den Stichen des
    Hochzeiters selber -- macht. In diesem Fall wird das Spiel als
    Normalspiel fortgesetzt. Beim Klärungsstich ist es unerheblich,
    ob eine Trumpfkarte oder eine Fehlfarbe ausgespielt wurde, so
    daß ein Aufspiel einer Herz Zehn eines Nichthochzeiters auch
    erlaubt ist.
  \item
    Der Hochzeiter macht die ersten drei Stiche selbst. jetzt wird
    der dritte Stich als Klärungsstich gezählt und das Spiel wird
    als "`Farbsolo in Karo"' fortgesetzt. Der Klärungsstich klärt
    daher auch, ob eine Hochzeit als Normalspiel oder als Farbsolo in
    Karo fortgesetzt wird. Der Hochzeiter wird im letzten Fall zum
    Alleinspieler und bildet die Re-Partei, die drei anderen Spieler
    bilden die Kontra-Partei. Das Spiel wird wie ein Solo
    abgerechnet, ersetzt jedoch nicht ein noch nicht gespieltes
    Pflichsolo, so daß es bei den Normalspielen notiert wird. Eine
    eventuelle Strafpunkteverteilung erfolgt bis einschließlich
    Klärungsstich wie beim Normalspiel, danach wie beim Solo. Diese
    Variante kann ein Hochzeiter auch anstreben, um An- bzw.
    Absagezeitpunke nach hinten zu verschieben.
\end{enumerate}
Bei einer angemeldeten Hochzeit ist die Erstansage erst erlaubt,
nachdem der Klärungsstich vollendet wurde (vierte Karte des
Stiches ist gespielt). Insoweit ist ein abweichender Zeitpunkt
gegenüber einem Normalspiel möglich, aber nicht zwangsläufig.

Stille Hochzeit:\\
Meldet ein Spieler, der beide Kreuz-Damen in der Hand hält,
keinen Vorbehalt bzw. einen solchen nicht regelgerecht an, spielt
er eine "`Stille Hochzeit"'. Das Spiel wird wie ein Solo
abgerechnet, ersetzt jedoch nicht ein noch nicht gespieltes
Pflichtsolo, so daß es bei den Normalspielen notiert wird. Eine
eventuelle Strafpunktverteilung erfolgt solange wie beim
Normalspiel, bis entschieden ist, daß ein Solo gespielt wird
(legen der zweiten Kreuz-Dame). An- und Absagezeitpunkte bei einer
"`Stillen Hochzeit"' sind wie beim Normalspiel. Es gilt nicht der
Sonderfall wie bei der angemeldeten Hochzeit.


\section{Spielverlauf}

\subsection{Aufspiel}
Aufspielpflicht zum ersten Stich hat der linke Nachbar des
Kartengebers. Es darf erst dann zum ersten Stich aufgespielt
werden, nachdem die Vorbehaltsfrage beendet und ein ggf.
ermittelter Vorbehalt getauft ist. Ab dem zweiten Stich spielt
immer derjenige auf, der den vorangegangenen Stich gemacht hat.
Eine gespielte Karte darf nicht zurückgenommen werden. Lediglich
bei gefordertem Weiterspiel nach einem Regelverstoß ist eine
regelgerechte Korrektur erlaubt.

\subsection{Bedienen}
Nach dem Aufspiel hat zunächst der linke Spieler des
Aufspielenden eine Karte zuzugeben. Dies geschieht, indem die
Karte offen auf den Spieltisch gelegt wird. Ebenso verhalten sich
im Uhrzeigersinn die beiden übrigen Spieler. Es besteht
Bedienpflicht, das heißt, jeder muß, wenn möglich eine Karte
der aufgespielten Fehlfarbe oder der geforderten Trumpffarbe
zugeben (bedienen). Wer die aufgespielte Fehlfarbe nicht hat,
darf entweder eine Trumpfkarte zugeben, das heißt, stechen, oder
eine beliebige Karte einer anderen Fehlfarbe spielen. Wenn eine
Trumpfkarte gefordert wird, aber nicht bedient werden kann, darf
eine beliebige Karte einer Fehlfarbe zugegeben werden. Eine
einmal gespielte Karte darf nur im Falle eines nicht reklamierten
Regelverstoßes zurückgenommen werden.

\subsection{Stiche}
Ein Stich besteht aus je einer Karte der vier Spieler. Er ist
vollendet, wenn jeder der vier Spieler in der vorgeschriebenen
Reihenfolge regelgerecht eine Karte gelegt hat, das heißt,
sobald die vierte Karte offen auf dem Spieltisch liegt und
niemand reklamiert. Die Positionen der Spieler werden "`im
laufenden Stich"' folgendermaßen bezeichnet:
\begin{center}
  \begin{tabular}{ll}
    Position 1 ist Vorhand & der Spieler, der die Aufspielpflicht für den Stich besitzt.\\
    Position 2 ist Mittelhand & der Spieler, der die 2. Karte des Stiches legen muß.\\
    Position 3 ist Mittelhand & der Spieler, der die 3. Karte des Stiches legen muß.\\
    Position 4 ist Hinterhand & der Spieler, der die 4. Karte des Stiches lagen muß\\
    & und damit den Stich beendet.\\
  \end{tabular}
\end{center}
Der Stich gehört demjeningen Spieler, der unter Beachtung der Regeln:
\begin{enumerate}
  \item zu einer aufgespielten und durchweg bedienten Fehlfarbe die ranghöchste Karte des
    Stiches gespielt hat,
  \item eine Fehlfarbe aufspielt, die weder bedient noch gestochen wird,
  \item eine aufgespielte Fehlfarbe als einziger sticht,
  \item eine Trumpfkarte auf eine aufgespielte Fehlfarbe mit einer ranghöheren Trumpfkarte nochmals übersticht.
  \item bei einer geforderten Trumpfkarte die ranghöchste Trumpfkarte des Stiches legt,
  \item Trumpfkarten fordert und darauf nur Karten von Fehlfarben bzw. rangniedrigere Trumpfkarten erhält.
\end{enumerate}
Wenn in einem Stich zwei gleiche Karten gelegt werden, ist die
vom weiter vorn sitzenden Spieler gelegte Karte die ranghöhere.
Dies gilt auch für die Herz Zehn.

Der Spieler, dem der Stich gehört, legt ihn als Ganzes verdeckt
vor sich hin. (Er darf allerdings Karten zur Erinnerung an
Sonderpunkte gesondert verdeckt (evtl. quer) ablegen, allerdings
muß hierdurch für alle die Rekonstruktion der Stiche und des
Spieles immer noch gewährleistet sein). Dieser Spieler hat die
Aufspielpflicht zum nächsten Stich. Die Stiche werden
übereinander abgelegt, nicht nebeneinander. Die Stiche sind so
zu vereinnahmen, daß jeder Spieler auch die zuletzt zugegebene
Karte deutlich erkennen kann. Der vorherige Stich ist auf
Verlangen eines Spielers solange einzusehen, bis die 4. Karte des
nächsten Stiches gespielt ist. Ausnahme: Gibt ein Spieler
rechtzeitg zu erkennen, den letzten Stich einsehen zu wollen, ist
dieser auch dann zu zeigen, wenn schon 4 Karten auf dem Tisch
liegen. jeder Stich ist einzeln einzuziehen, folgerichtig
abzulegen und bis Spielende verdeckt nachprüfbar zu belassen.
Falls der Solospieler einen oder mehrere Stiche hintereinander
nicht einzieht, fallen diese und alle folgenden Stiche an die
Kontra-Partei, sobald ein Stich abgegeben wird. Das Nachsehen,
Nachzählen oder Aufdecken der abgelegten Stiche bzw. Nachzählen
der Augen durch einen Spieler beendet das Spiel.

\subsection{Spielabkürzung}
Im allgemeinen ist jedes Spiel zu Ende zu spielen! Die
Frage:"`Darf ich abkürzen?"', oder eine ähnliche ist hier rein
rhetorisch zu verstehen, das heißt, sie impliziert die
Abkürzung. Eine Spielabkürzung ist grundsätzlich nur einem
Solospieler gestattet. Durch das Auflegen oder Vorzeigen seiner
Karten während des Spieles ohne Angabe einer Erklärung zeigt
der Solospieler an, daß er alle weiteren Stiche macht. Falls der
Solospieler zu diesem Zeitpunkt kein Aufspiel hat, ist davon
auszugehen, daß er mit beliebiger Karte übernimmt bzw. jede
Fehlfarbe, die er nicht besitzt mit beliebiger Trumpfkarte sticht.
Wenn der Solospieler die Stiche nur dann macht, falls er seine
Karten in einer bestimmten Reihenfolge spielt, muß er diese
Reihenfolge unaufgefordert angeben. Es ist allerdings davon
auszugehen, daß der Solospieler sowohl in den Trumpfkarten als
auch in den Fehlfarben von oben spielt. Allerdings muß nicht
davon ausgegangen werden, daß er eine Trumpfkarte zuerst spielt,
wenn er dieses nicht ausdrücklich beim Abkürzen angibt.

Der Solospieler darf das Spiel jederzeit abkürzen, indem er
seine nicht gespielten Karten der Gegenpartei übergibt. Die
Gegenpartei macht somit alle Reststiche. Der Solospieler verwirkt
damit das Recht zur Reklamation von zuvor erfolgten
Regelverstößen. Kürzt der Solospieler ab und der Zeitpunkt
für eine An- bzw. Absage ist noch nicht vorbei, ist es der
Gegenpartei erlaubt, die mögliche Ansage bzw. die möglichen
Absagen noch zu machen. Die restlichen Karten werden jedoch nicht
mehr gespielt.

\subsection{Klärung der Parteizugehörigkeit und Partnerschaft}
Die Parteizugehörigkeit eines Spielers kann nur auf folgende
Weise zweifelsfrei eindeutig geklärt werden:
\begin{enumerate}
  \item Durch die Bekanntgabe eines Solos,
  \item bei einer Hochzeit durch den Klärungsstich,
  \item beim Normalspiel und bei der "`Stillen Hochzeit"' durch gelegte Kreuz-damen,
  \item beim Normalspiel durch An- oder Absagen,
  \item durch Abwurf einer Fehlkarte bei Anspiel einer Trumpfkarte beim Normalspiel oder bei einer "`Stillen Hochzeit"'.
\end{enumerate}
Die Partnerschaften sind erst dann eindeutig geklärt, wenn nach
den Kriterien für alle vier Spieler die Parteizugehörigkeit
eindeutig geklärt ist. Das legen einer einzelnen Kreuz-Dame bzw.
lediglich eine Ansage klären die Partnerschaft nicht in diesem
Sinne.


\section{Ansagen und Absagen}

\subsection{Definitionen}
Eine Karte gefindet sich "`in der Hand"' eines Spielers, wenn sie
noch nicht gespielt ist, das heißt ein Karte ist gespielt, wenn
sie den Spieltisch offen (sichtbar) berührt hat. Das Wort
"`mindestens"' gibt an, daß eine An- bzw. Absage vor dem
spätest möglichen Zeitpunkt gemacht werden darf, ohne daß sich
dadurch die nachfolgenden Absagen oder Erwiderungen auf einen
entsprechend vorgezogenen Zeitpunkt verschieben müssen.

\subsection{Zweck von Ansagen und Absagen}
Ansagen und Absagen sowie deren Zeitpunkt können im Doppelkopfspiel mehrere Bedeutungen haben:
\begin{enumerate}
  \item Sie erhöhen den Spielwert.
  \item Sie können die Parteienfindung beeinflussen: Mit einer Ansage "`Re"' kann ein Spieler anzeigen, daß er zur Re-Partei gehört.
    Mit einer Ansage "`Kontra"' kann ein Spieler anzeigen, daß er zur Kontra-Partei gehört.
  \item Sie können das Spiel steuern:
    Mit einer An- bzw. Absage vor dem letztmöglichen Zeitpunkt kann ein Spieler versuchen, besondere Stärken seines
    Blattes anzuzeigen oder seinen Partner auf ein gewünschtes Weiterspiel hinzuweisen.
\end{enumerate}

\subsection{Ansagen und Ansagezeitpunkt}
Eine Erstansage ist frühestens möglich, sobald geklärt ist,
welcher Spieltyp gespielt wird. Das bedeutet, daß die
Vorbehaltsfrage beendet sein muß, ggf. ein Solo getauft sein
muß und bei einer Hochzeit der Klärungsstich beendet sein muß.
Die Erstansage von "`Re"' (mit den Kreuz-Damen bzw. Solospieler)
oder "`Kontra"' (gegen die Kreuz-Damen bzw. gegen den
Solospieler) muß von jeder Partei spätestens dann erfolgen,
wenn der ansagende Spieler noch mindestens 11 Karten in der Hand
hält. Der erste Stich wird daher auch als "`Freistich"'
bezeichnet. Eine Ausnamhe bezüglich des Ansagezeitpunktes bildet
die angemeldete Hochzeit.

\subsection{Absagen und Absagezeitpunkt}
Beide am Spiel beteiligten Parteien haben die Möglichkeit, der
Gegenpartei abzusagen, daß diese eine bestimmte Augenzahl nicht
erreichen wird. Die im Spiel befindlichen 240 Augen sind hierzu
in Stufen (sogenannte Limits) eingeteilt. Jede Stufe umfaßt 30
Augen. Die Absage "`keine 90"' ("`keine 60"', "`keine 30"')
behauptet, daß die Gegenpartei unter der genannten Augenzahl
bleiben wird. Die Absage "`schwarz"' behauptet, daß die
Gegenpartei keinen Stich machen wird. Wenn die Gegenpartei jedoch
das abgesagte Ziel (die jeweilige Augenzahl bzw. den Stich)
erreicht, hat die ansagende Partei nicht gewonnen, auch wenn sie
mehr als 120 Augen erspielt hat. Eine Absage ist zulässig, wenn
vorher die eigene Partei eine regelgerechte Erstansage gemacht
hat. Durch jede Absage wird der Spielwert zusätzlich erhöht. Es
sei denn, keine Partei erreicht das von ihr abgesagte Ziel. Für
den jeweiligen Absagezeitpunkt gilt:
\begin{itemize}
  \item "`keine 90"' mit mindestens 10 Karten in der Hand,
  \item "`keine 60"' mit mindestens 9 Karten in der Hand,
  \item "`keine 30"' mit mindestens 8 Karten in der Hand,
  \item "`schwarz"' mit mindestens 7 Karten in der Hand.
\end{itemize}
Eine Ausnahme bezüglich des Absagezeitpunktes bildet die angemeldete Hochzeit.

\subsection{Ergänzungen zu den An- und Absagezeitpunkten}
Unter Beachtung des frühest und des spätest möglichen
Zeitpunktes für An- bzw. Absagen ist eine An- bzw. Absage auch
regelgerecht, wenn der betreffende Spieler nicht mit dem Legen
einer Karte an der Reihe ist.

Die angemeldete Hochzeit:\\
Eine Ausnahme bezüglich des Zeitpunktes für An- und Absagen
gilt bei einer angemeldeten Hochzeit. Falls der erste Stich der
Klärungsstich ist, verschieben dich die genannten Ansage- bzw.
Absagezeitpunkte nicht. Wenn der zweite (dritte) Stich der
Klärungsstich ist, verringert sich die dort genannte Kartenzahl
um 1(2). Eine Erstansage ist immer erst nach Beendigung des
Klärungsstiches erlaubt!
\\
Wenn der zulässige Zeitpunkt für eine Erstansage bzw. eine
mögliche Absage nicht überschritten ist, dürfen beliebig viele
Stufen übersprungen werden, daß heißt, beim "Uberspringen von
An- bzw. Absagen müssen diese nicht genannten An- bzw. Absagen
zu eben diesem Zeitpunkt noch zulässig sein. Es werden in diesem
Fall alle Stufen in der Abrechnung mit einbezogen. daraus folgt,
daß es unzulässig ist, eine unterlassene An- bzw. Absage nach
dem spätest möglichen Zeitpunkt durch die nächste Absage
nachzuholen. Die Erwiderung auf eine An- bzw. Absage ist stets
noch mit einer Karte weniger, als für die An- bzw. Absage der
Gegenpartei notwendig war, erlaubt. Als Erwiderung ist nur
"`Kontra"' (gegen die Re-Partei) bzw. "`Re"' (gegen die
Kontra-Partei) zulässig. Eine Absage der eigenen Partei danach
ist nur dann regelgerecht, wenn die eigene Erwiderung auch als
Ansage rechtzeitig erfolgt wäre. Wenn "`Kontra"' und "`Re"'
(gleichgültig, in welcher Reihenfolge) angesagt werden, muß ein
dritter Spieler, bzw. vierter Spieler, falls er bei ungeklärter
Partnerschaft eine Absage vornimmt, zu erkennen geben, ob er zur
Kontra- oder zu Re-Partei gehört. Es ist zulässig, daß beide
Parteien "`keine 90"' ("`keine 60"' usw.) absagen. Falls mehrere
Spieler gleichzeitg oder nacheinander verschiedene An- oder
Absagen machen, sind alle gültig. Wenn ein Spieler noch einmal
auf seine eigene An- bzw. Absage hinweist, wenn er meint, daß
die anderen Spieler sie infolge von Lärm nicht gehört haben,
ist dies regelgerecht. Es liegt damit keine unzulässige
Wiederholung einer An- bzw. Absage vor.


\section{Spielbewertung}

\subsection{Gewinnstufen und Gewinnkriterien}
Die Fälle, in denen eine Partei gewonnen hat und den Grundwert von einem Spielpunkt dafür erhält,
werden im folgenden vollständig aufgeführt, wobei nur bereits vollständige
und regelgerechte Stiche berücksichtigt werden.
Die Re-Partei hat in den folgenden Fällen gewonnen:
\begin{enumerate}
  \item Mit dem 121. Auge, wenn keine Ansage bzw. Absage getroffen wurde,
  \item mit dem 121. Auge, wenn nur "`Re"' angesagt wurde,
  \item mit dem 121. Auge, wenn "`Re"' und "`Kontra"' unabhängig von der Reihenfolge angesagt wurden,
  \item mit dem 120. Auge, wenn nur "`Kontra"' angesagt wurde,
  \item mit 151 (181., 211.) Auge, wenn sie der Kontra-Partei "`keine 90"' ("`keine 60"', "`keine 30"') abgesagt hat,
  \item falls sie alle Stiche erhalten hat, wenn sie der Kontra-Partei "`schwarz"' abgesagt hat,
  \item mit dem 90. (60., 30.) Auge, wenn von der Kontra-Partei "`keine 90"' ("`keine 60"', "`keine 30"') abgesagt
    wurde und die Re-Partei sich nicht durch eine Absage zu einer höheren Augenzahl verpflichtet hat,
  \item mit dem 1. Stich, den sie erhält, wenn von der Kontra-Partei "`schwarz"' abgesagt wurde und die Re-Partei
    sich nicht durch eine Absage zu einer höheren Augenzahl verpflichtet hat.
\end{enumerate}
Die Kontra-Partei hat in den folgenden Fällen gewonnen:
\begin{enumerate}
  \item Mit dem 120. Auge, wenn keine Ansage bzw. Absage getroffen wurde,
  \item mit dem 120. Auge, wenn nur "`Re"' angesagt wurde,
  \item mit dem 120. Auge, wenn "`Re"' und "`Kontra"' unabhängig von der Reihenfolge angesagt wurden,
  \item mit dem 121. Auge, wenn nur "`Kontra"' angesagt wurde,
  \item mit 151 (181., 211.) Auge, wenn sie der Re-Partei "`keine 90"' ("`keine 60"', "`keine 30"') abgesagt hat,
  \item falls sie alle Stiche erhalten hat, wenn sie der Re-Partei "`schwarz"' abgesagt hat,
  \item mit dem 90. (60., 30.) Auge, wenn von der Re-Partei "`keine 90"' ("`keine 60"', "`keine 30"') abgesagt
    wurde und die Kontra-Partei sich nicht durch eine Absage zu einer höheren Augenzahl verpflichtet hat,
  \item mit dem 1. Stich, den sie erhält, wenn von der Re-Partei "`schwarz"' abgesagt wurde und die Kontra-Partei
    sich nicht durch eine Absage zu einer höheren Augenzahl verpflichtet hat.
\end{enumerate}

\subsection{Spielwerte}
Die Spielwerte der Einzelspiele werden in Spielpunkten (Punkten)
ausgedrückt. Es wird nach der PLUS-MINUS-Wertung gewertet: Beim
Normalspiel erhalten die Spieler der Siegerpartei folgende
Spielpunkte mit positivem, die Spieler der Verliererpartei mit
negativen Vorzeichen:
\begin{enumerate}
  \item Gewonnen: 1 Punkt als Grundwert
    \begin{enumerate}
      \item unter 90 gespielt: 1 Punkt zusätzlich
      \item unter 60 gespielt: 1 Punkt zusätzlich
      \item unter 30 gespielt: 1 Punkt zusätzlich
      \item schwarz gespielt: 1 Punkt zusätzlich
    \end{enumerate}
  \item Es wurde:
    \begin{enumerate}
      \item "`Re"' angesagt: 2 Punkte zusätzlich
      \item "`Kontra"' angesagt: 2 Punkte zusätzlich
    \end{enumerate}
  \item Es wurde von der Re-Partei:
    \begin{enumerate}
      \item "`keine 90"' abgesagt: 1 Punkt zusätzlich
      \item "`keine 60"' abgesagt: 1 Punkt zusätzlich
      \item "`keine 30"' abgesagt: 1 Punkt zusätzlich
      \item "`schwarz"' abgesagt: 1 Punkt zusätzlich
    \end{enumerate}
  \item Es wurde von der Kontra-Partei:
    \begin{enumerate}
      \item "`keine 90"' abgesagt: 1 Punkt zusätzlich
      \item "`keine 60"' abgesagt: 1 Punkt zusätzlich
      \item "`keine 30"' abgesagt: 1 Punkt zusätzlich
      \item "`schwarz"' abgesagt: 1 Punkt zusätzlich
    \end{enumerate}
  \item Es wurden von der Re-Partei:
    \begin{enumerate}
      \item 120 Augen gegen "`keine 90"' erreicht: 1 Punkt zusätzlich
      \item 90 Augen gegen "`keine 60"' erreicht: 1 Punkt zusätzlich
      \item 60 Augen gegen "`keine 30"' erreicht: 1 Punkt zusätzlich
      \item 30 Augen gegen "`schwarz"' erreicht: 1 Punkt zusätzlich
    \end{enumerate}
  \item Es wurden von der Kontra-Partei:
    \begin{enumerate}
      \item 120 Augen gegen "`keine 90"' erreicht: 1 Punkt zusätzlich
      \item 90 Augen gegen "`keine 60"' erreicht: 1 Punkt zusätzlich
      \item 60 Augen gegen "`keine 30"' erreicht: 1 Punkt zusätzlich
      \item 30 Augen gegen "`schwarz"' erreicht: 1 Punkt zusätzlich
    \end{enumerate}
\end{enumerate}
Sonderpunkte können von beiden Parteien nur beim Normalspiel gewonnen werden.
Sie werden ggf. zuerst miteinander und danach mit den unter (1 - 6) ermittelten
Punkten verrechnet.
Es werden folgende Sonderpunkte vergeben:
\begin{itemize}
  \item Gegen die Kreuz-Damen gewonnen: 1 Sonderpunkt
  \item Doppelkopf (ein Stich mit 40 oder mehr Augen): 1 Sonderpunkt
  \item Karo As (Fuchs) der Gegenpartei gefangen: 1 Sonderpunkt
  \item Kreuz Bube (Karlchen) macht den letzten Stich: 1 Sonderpunkt
\end{itemize}
Bei einem Solo gibt es keine Sonderpunkte. Dies gilt auch für
eine "`Stille Hochzeit"'. Diese Punktzahl wird für den
Solospieler verdreifacht und ihm bei Gewinn gutgeschrieben, bei
Niederlage abgezogen. Den drei Spielern der Gegenpartei wird die
einfache Punktzahl mit umgekehrtem Vorzeichen angeschrieben.

\subsection{Spielliste}
Jedes Spiel ist unmittelbar nach seiner Beendigung als Gewinn
oder Verlust für die beteiligten Spieler in die Spielliste
einzutragen. Für die korrekte Notation ist nicht nur der
Listenführer, sondern sind alle Spieler gleichermaßen
verantwortlich. Daher wird empfohlen, daß der jeweils aktuelle
Kartengeber die Eintragungen in der Spielliste auf ihre
Richtigkeit überprüft. Korrekte Notation bedeutet, daß die
richtige Punktzahl bei den richtigen Spielern mit dem richtigen
Vorzeichen an der richtigen Stelle auf dem Spielzettel eingetragen
wird. Um hier Fehler frühzeitig zu erkennen, empfiehlt es sich,
jeweils die folgenden Punkte zu überprüfen:
\begin{enumerate}
  \item Die Quersumme muß nach jedem Spiel Null ergeben.
  \item die Spielerergebnisse der 4 Mitspieler sind nach jedem Spiel entweder alle gerade oder ungerade (Gerade-Ungerade-Regel).
    Diese Regel kann nur durch die Vergabe von Strafpunkten in einem Solo zugunsten des Solospielers verletzt werden.
  \item die jeweiligen Spieler, die in einem Spiel positive Punkte erreicht haben, müssen gekennzeichnet sein, indem
    ihre Positionsnummern notiert werden oder indem diese Spieler
    eindeutig markiert werden.
\end{enumerate}
Der ausgeloste Listenführer ist dafür verantwortlich, daß die Spielliste nach Beendigung der Runde bei der
Turnierleitung abgegeben wird.


\section{Turnier-Doppelkopf}

\subsection{Offizielle Turniere}
Alle offiziellen Turniere des DDV sind unter Beachtung dieser Turnierspielregeln durchzuführen.

\subsection{Schiedsrichter, Schiedsgericht und Ersatzschiedsgericht}
Es werden ein Schiedsrichter, ein Schiedsgericht und ein
Ersatzschiedsgericht eingesetzt. Im Falle einer Reklamation
entscheidet der Schiedsrichter in erster Instanz über die
Vergabe von Strafpunkten. Das Schiedsgericht wird einberufen,
wenn ein Spieler mit einer Entscheidung des Schiedsrichters nicht
einverstanden ist und bei Verdacht auf Unsportlichkeit. Die
Entscheidung des Schiedsgerichtes ist endgültig. Ein Mitglied
des Ersatzschiedsgerichtes kommt zum Einsatz, wenn ein Mitglied
des Schiedsgerichtes selbst direkt von der Entscheidung betroffen
ist. Vergebene Strafpunkte sind vom Schiedsrichter oder von einem
Mitglied des Schiedsgerichts in jeden Fall abzuzeichnen.

\subsection{Vierer-Tische}
Es wird ausschließlich an Vierer-Tischen gespielt.

\subsection{Sitzreihenfolge}
Die Sitzreihenfolge am Tisch wird durch einen Turnierplan
festgelegt, der vor dem Turnier durch die Turnierleitung bekannt
gegeben wird. An jedem Tisch gilt, daß der Spieler auf Position
1 als erster gibt und der Spieler auf Position 4 die Spielliste
führt. Wenn sich ein anderer Spieler bereit erklärt, die
Spielliste zu führen, hat dies keinen Einfluß auf die
Sitzreihenfolge.

\subsection{Anzahl der Spiele und Spielzeit}
Eine Spielrunde besteht aus 24 Spielen. Sie beginnt stets erst
mit der offiziellen Freigabe durch die Turnierleitung bzw. durch
den Schiedsrichter. Zuvor darf lediglich die Spielvorbereitung
abgeschlossen sein, und es dürfen die Karten aufgenommen werden.
Der Beginn der Vorbehaltsfrage ist vor Freigabe der Runde
unzulässig.

Die reine Spielzeit beträgt für alle Tische gleichermaßen 100
Minuten. In begründeten Ausnahmefällen darf die Turnierleitung
für einzelne Tische hiervon abweichen. Das Ende der Spielrunde
wird durch die Turnierleitung bzw. durch den Schiedsrichter
bekannt gegeben. Danach werden angefangene Spiele (mit Beginn des
Austeilens gilt ein Spiel als angefangen) zu Ende gespielt.
Ausstehende Normalspiele werden durch die Turnierleitung bzw. den
Schiedsrichter oder das Schiedsgericht gestrichen. Wenn auf der
Spielliste eine Nachspielzeit durch den Schiedsrichter notiert
ist, verlängert sich die offizielle Spielzeit dieses Tisches um
die dort angegebene Nachspielzeit.

Eine Spielrunde ist beendet, wenn kein Mitspieler mehr die
errechneten Spielpunkte überprüfen oder beanstanden will. Nach
"Uberschreiten der Spielzeit werden mit Ausnahme von vorgeführten
Pflichtsoli keine Spiele wiederholt. Dieses gilt auch für Spiele
mit falsch ausgeteilten Karten.

\subsection{Pflichtsolo}
In jeder Spielrunde ist von jedem Spieler ein Pflichtsoli zu
spielen. Ist das für die Spielrunde festgesetzte Zeitlimit
erreicht, und es stehen noch Pflichtsoli aus, müssen die
betreffenden Spieler ihr Pflichtsolo noch spielen (das heißt,
sie werden "`vorgeführt"').


\section{Regelverstöße und Strafpunkte}

\subsection{Allgemeine Grundregeln}
Als oberstes Spielgebot gilt, die einzelnen Punkte der
Turnierspielregeln des DDV auch zur weiteren Förderung des
Doppelkopfspieles nach Verbandsregeln zu beachten und einzuhalten.
Alle Teilnehmer haben in jeder Situation das Prinzip der Fairness
und Sachlichkeit zu wahren und kein fadenscheiniges Recht zu
suchen.Die Wiederholung eines Spieles ist normalerweise ohne das
Vorliegen und Ahnden eines schwerwiegenden Regelverstoßes nicht
zulässig, da dieses Regeln in (fast) allen Fällen eine
nachvollziehbare Entscheidung ermöglichen. In Ausnahmefällen,
in denen in jedem Fall auch das Schiedsgericht hinzuzuziehen ist,
ist der Sachverhalt der zur Spielwiederholung führte, zu
protokollieren und unverzüglich der Regelkommission des DDV
zuzuleiten. Lautes Zählen der Trumpfkarten oder Augen ist weder
den Spielern noch den anderen Teilnehmern oder Zuschauern erlaubt.
Spieler, Teilnehmer und Zuschauer haben sich jeglicher
"Außerungen und Gesten zu enthalten, die geeignet sind, die
Karten zu verraten oder den Spielverlauf zu beeinträchtigen bzw.
zu beeinflussen.

Das Animieren eines Spielers zu einem Regelverstoß stellt eine
Unsportlichkeit dar. Falls der Regelverstoß dann allerdings
tatsächlich begangen wird, müssen Unsportlichkeit und
Regelverstoß entsprechend geahndet werden, d.h. in der Regel
erhalten dann zwei Spieler Strafpunkte.

Wenn für einen Regelverstoß nicht ausdrücklich in diesen TSR
bzw. im Anhang der TSR bzw. durch die Regelkommision geregelt
ist, ob er als unerhebliche, geringfügiger oder schwerwiegender
Regelverstoß bzw. als unsportliches Verhalten zu klassifizieren
ist, ist die Klassifikation nach nachfolgenden Kriterien
vorzunehmen.

\subsection{Unerhebliche Regelverstöße}
Ein "`unerheblicher Regelverstoß"' liegt dann vor, wenn der
Regelverstoß keinen Einfluß auf den weiteren Spielverlauf bzw.
Spielausgang hat. Wenn eine Partei zwangsläufig alle Reststiche
macht, ist jeder reklamierte Regelverstoß unerheblich.
Zwangsläufig in diesem Zusammenhang bedeutet, daß die
betreffende Partei entweder alle Reststiche -- unabhängig davon,
in welcher Reihenfolge die eigenen Karten gelegt werden -- macht,
oder daß die Partei vor einem evtl. Regelverstoß die korrekte
Reihenfolge angegeben hat, in der die eigenen Spielkarten gelegt
werden müssen, damit die eigene Partei alle Reststiche macht.
Zwangsläufig bedeutet jedoch nicht bei beliebiger
Kartenverteilung, sondern bei der gegebenen.

Wenn ein Spieler einen unerheblichen Regelverstoß begeht, werden
keine (0) Strafpunkte verteilt. Das Spiel wird nach Erledigung
der Reklamation fortgesetzt.

\subsection{Geringfügige Regelverstöße}
Ein "`geringfügiger Regelverstoß"' liegt dann vor, wenn durch
einen Regelverstoß kein entscheidender Einfluß auf den Sieger
des Spieles gegeben ist, das heißt, eine spielentscheidende
Bevorteilung der eigenen Partei ausgeschlossen ist. Der
Regelverstoß kann aber sehr wohl Einfluß auf den Spielverlauf
und -ausgang haben.

\subsubsection{Normalspiel}
Begeht ein Spieler einen geringfügigen Regelverstoß
beim Normalspiel, erhält er im Falle einer Reklamation 3
Strafpunkte. Die drei Mitspieler erhalten je einen Punkt
gutgeschrieben. das Spiel wird nach Erledigung der Reklamation
fortgesetzt.

\subsubsection{Solo}
Begeht ein Spieler einen geringfügigen Regelverstoß,
nachdem feststeht, daß ein Solo gespielt wird, wird
folgendermaßen verfahren:
\begin{enumerate}
  \item
    Bei einen geringfügigen Fehler des Solospielers erhält dieser
    im Falle einer Reklamation 3 Strafpunkte. Die drei Mitspieler
    erhalten je einen Punkt gutgeschrieben. Das Spiel wird nach
    Erledigung der Reklamation fortgesetzt.
  \item
    Bei einem geringfügigen Regelverstoß eines Spielers der
    Kontra-Partei werden zwei Fälle unterschieden:
    \begin{enumerate}
      \item
	Der Regelverstoß kann zu einem Nachteil für den Solospieler
	führen: Der Verursacher erhält im Falle einer Reklamation 3
	Strafpukte, die dem Solospieler gutgeschrieben werden. Die beiden
	übrigen Mitspieler erhalten 0 Punkte. Das Spiel wird nach
	Erledigung der Reklamation fortgesetzt.
      \item
	Der Regelverstoß kann nicht zu einem Nachteil für den
	Solospieler führen: Der Verursacher erhält im Falle einer
	Reklamation 3 Strafpunkte. Die drei Mitspieler erhalten je einen
	Punkte gutgeschrieben. das Spiel wid nach Erledigung der
	Reklamation fortgesetzt.
    \end{enumerate}
\end{enumerate}

\subsection{Schwerwiegende Regelverstöße}
Ein "`schwerwiegender Regelverstoß"' liegt dann vor, wenn der
Regelverstoß einen entscheidenden Einfluß auf den Sieger des
Spieles haben kann, das heißt, eine spielentscheidene
Bevorteilung der eigenen Partei möglich ist. Wenn ein Spieler
einen schwerwiegenden Regelverstoß begeht, ist im Falle einer
Reklamation das Spiel sofort  zu beenden. In diesen Fällen
werden folgende Abläufe bei der Vergabe von Strafpunkten
unterschieden:

\subsubsection{Normalspiel}
Hier wird das vorzeitig beendete Spiel als "`nicht
gespielt"' gewertet. Der Verursacher erhält 12 Strafpunkte.
Diese Strafpunktzahl erhöht sich für den Verursacher um je 3
Punkte für jede Absage, die die Gegenpartei bis zum Eintritt des
Regelvestoßes gemacht hat, die drei Mitspieler erhalten je 4
(bzw. 5, 6, 7 oder 8) Pluspunke.

\subsubsection{Solo}
Bei einem schwerwiegenden Regelverstoß des Solospielers
wird das Spiel als nicht gespielt gewertet. Hingegen wird bei
einem schwerwiegenden Regelverstoß eines Spielers der
Gegenpartei das Solo als gespielt gewertet. Der Verursacher
erhält 12 Minuspunkte. Diese Punktzahl erhöht sich für den
Verursacher um je 3 Punkte für jede Absage, die der Solospieler
bis zum Eintritt des Regelverstoßes gemacht hat. Der Solospieler
erhält 12 (bzw. 15, 18, 21, 24) Pluspunkte. Die übrigen Spieler
erhalten 0 Punkte.

\subsection{Regelverstöße bei bereits entschiedenem Spiel}
Wenn ein Regelverstoß begannen und reklamiert wird, der eine
korrekte Weiterführung des Spieles nicht zuläßt, und eine
Partei hat das Spiel nach An- und Absagen bereits gewonnen,
erfolgt ein Spielabbruch. Die Reststiche ab dem Zeitpunkt des
Reglverstoßes gehen an die Gegenpartei. Weitere Sonderpunte
können nicht mehr erzielt werden. Der Regelverstoß wird
zusätzlich wie ein geringfügiger Regelverstoß geahndet. Zur
Ermittlung, ob ein Spiel bereits entschieden ist, zählen nur
bereits vollständig und regelgerecht beendete Stiche.

\subsection{Unsportliches Verhalten}
Ein unsportliches Verhalten liegt dann vor, wenn ein Spieler mit
nicht spielerischen Mitteln versucht, seine Spieler zu
beeinflussen oder die regelgerechte Wertung eines Spieles oder
einer Spielrunde zu verhindern. Bei Verdacht auf unsportliches
Verhalten muß der Schiedsrichter das Schiedsgericht einberufen.
Das Schiedsgericht kann, je nach Schwere des Deliktes, die
folgenden Sanktionen verhängen:
\begin{enumerate}
  \item Verwarnung (führt bei weiterem Fehlverhalten zu Punktabzug oder zu Disqualifikation),
  \item Punktabzug (maximal 12), ohne Gutschrift für die Mitspieler,
  \item Disqualifikation.
\end{enumerate}

\subsection{Reklamation}
Das Recht, einen Regelverstoß zu reklamieren, haben bei
geklärter Partnerschaft nur die Gegenspieler, bei ungeklärter
Partnerschaft dagegen alle am Spiel beteiligten Spieler, mit
Ausnahme des Verursachers. Ungeklärte Partnerschaft liegt vor,
wenn die Pateizugehörigkeit noch nicht für alle vier Spieler
zweifelsfrei geklärt ist. Jeder Spieler darf auf einen
Regelverstoß hinweisen. Ein Regelverstoß ist sofort, nachdem er
von allen Spielern entdeckt werden konnte, zu reklamieren. Wer
weiterspielt, verliert das Anrecht auf Reklamation. Eine
Spielmaßnahme (Fortsetzung des Vorbehaltsfrage, Ansage, Absage,
Bemerkung oder Legen einer Karte), die einen Regelverstoß
darstellt, ist als gültiger Spielzug anzusehen, wenn sie nicht
rechtzeitig reklamiert wurde. Als Weiterspielen werden die
Fortsetzung der Vorbehaltsfrage (Gesund-, Vorbehaltsmeldung oder
Taufen eines Vorbehaltes), das Legen einer Karte bzw. eine An-
oder Absage gewertet. Eine Anfrage an den/die
reklamationsberechtigten Partner, ob reklamiert oder ob
weitergespielt werden soll, ist erlaubt. Der/Die Parteien
darf/dürfen nur mit "`Reklamation"' bzw. "`keine Reklamation"'
antworten. Seine/ihre Antwort ist bindend, das heißt, ihm/ihnen
wird die Entscheidung überlassen! Im Falle der Reklamation wird
je nach Schwere des Fehlers verfahren. Im Falle der
Nichtreklamation wird der Regelverstoß als gültiger Spielzug
angesehen. Bei jeder Reklamation ist der Schiedsrichter zu rufen.
Das Spiel ruht sofort, das heißt, der laufende Stich bleibt
offen, die anderen Stiche und die restlichen Karten bleiben
verdeckt liegen, und es werden keine Diskussionen über das Spiel
geführt. Wenn falsch gegeben wurde, kann sofort erneut gegeben
werden. Hierzu braucht also keine Schiedsrichterentscheidung
abgewartet werden. Es wird der zuerst reklamierte Regelverstoß,
also nicht notwendigerweise der zuerst begangene Regelverstoß
bestraft. Bei gleichzeitiger Reklamation verschiedener
Regelverstöße wird der zuerst begangene bestraft, sofern dieser
noch reklamiert werden darf. Wenn dann der erste kein
schwerwiegender Regelverstoß ist, kann der zweite Regelverstoß
erneut reklamiert werden. Gegen die Entscheidung des
Schiedsrichters kann nur sofort reklamiert werden, das heißt,
daß das Spiel bis zur Entscheidung des Schiedsgerichtes
unterbrochen wird. Die Spielzeit des betroffenen Tisches ggf.
auch der Tische, an denen die Mitglieder des Schiedsgerichtes
spielen, wird um die Zeit der Unterbrechung verlängert.


\section{Abweichunen und Empfehlungen}
Abweichend kann Doppelkopf auch von 5, 6 oder 7 Spielern gespielt
werden. Es wird dann an Fünfer-, Sechser- oder Siebener-Tischen
gespielt. Statt des Spieles an einem Siebener-Tisch kann ein
Spieler ("`Springer"') auch an zwei Tischen gleichzeitig spielen.

\subsection{Abweichungen an Fünfer-Tischen}
\begin{itemize}
  \item Abweichend erhält bei fünf Mitspielern der Kartengeber selbst keine Karten; er "`sitzt"'.
  \item Der Geber erhält am Fünfer Tisch weder Plus- noch Minuspunkte.
  \item Abweichend muß am Fünfer-Tisch die Gerade-Ungerade-Regel nicht erfüllt sein.
  \item Abweichend führt am Fünfer-Tisch der Spieler an Position 5 die Spielliste.
  \item Abweichend dauert eine Spielrunde am Fünfer-Tisch 30 Spiele und ist nach einer reinen Spielzeit von 125 Minuten zu beenden.
\end{itemize}

\subsection{Empfehlung für Fünfer-Tische}
Entsprechend obiger Regeln werden nach einem Pflichtsolo die Karten nochmals vom selben Geber verteilt, so daß im Extremfall ein Spieler nur an 21 Spielen beteiligt ist, während die anderen 24 bzw. 25 absolvieren. Jeder Spieler hat jedoch in jedem Fall bei genau fünf Normalspielen Aufspielpflicht, so daß niemand an mehr als 25 Spielen beteiligt sein kann.  

\section{Inkrafttreten}
Diese Turnierspielregeln gelten ab dem 1. Juli 2000\\
Essen, den 19 Februar 2000\\
Deutscher Doppelkopf-Verband e.V. (DDV)\\
-- Die Regelkommission des DDV --
\end{document}
